\documentclass{report}
\usepackage[french]{babel}

% Permet d'ajuster la taille des marges et de la distance pour les footer
\usepackage[tmargin=2cm,rmargin=1in,lmargin=1in,margin=0.85in,bmargin=2cm,footskip=.2in]{geometry}

% Permet d'optimiser l'affichage de différents symboles et formules mathématiques
\usepackage{amsmath,amsfonts,amsthm,amssymb,mathtools}

\usepackage{svg}

% Modifie l'apparence des nombre en mathmode et textmode
\usepackage[varbb]{newpxmath}

% Modifier l'apparence des fractions
\usepackage{xfrac}

% Permet de rayer (barrer) l'argument avec la touche
% \cancel{} \bcancel{} ou \xcancel{}
\usepackage[makeroom]{cancel}

% Extension du package amsmath; corrige certains bugs et déficiences de son prédecesseur
\usepackage{mathtools}

% This package provides most of the flexibility you may want to customize the three basic list
% environments (enumerate, itemize and description)
\usepackage{bookmark} 

% Réorganiser les théorèmes et Lemmes. Usage complexe. 
% Référence : https://ctan.math.illinois.edu/macros/latex/contrib/theoremref/theoremref-doc.pdf
\hypersetup{hidelinks}
\usepackage{hyperref,theoremref} 

% Fournit un environnement pour créer des boîtes colorées
\usepackage[most,many,breakable]{tcolorbox}


%\newcommand\mycommfont[1]{\footnotesize\ttfamily\textcolor{blue}{#1}}\SetCommentSty{mycommfont}

%\newcommand{\incfig}[1]{%\def\svgwidth{\columnwidth}\import{./figures/}{#1.pdf_tex}}
\newcommand{\arc}[1]{\wideparen{#1}}

%Pour colorer les lignes séparatrices de tableaux
\usepackage{colortbl}
\usepackage{tikzsymbols}

\usepackage{framed}
\usepackage{titletoc}
\usepackage{etoolbox}
\usepackage{lmodern}
\usepackage{tabularx}
\usepackage{enumitem}
\usepackage{amsthm}
%==========================================================================================
\usepackage{libris} 
\usepackage{etoolbox}
\usepackage[export]{adjustbox}% for positioning figures

\makeatletter
% Force le chapitre suivant sur la ligne succedant la fin du 
% chapitre précédent
\patchcmd{\chapter}{\if@openright\cleardoublepage\else\clearpage\fi}{}{}{}
\makeatother
\usepackage[Glenn]{fncychap}


%boîte de couleur grise
\tcbset{
  graybox/.style={
    colback=gray!20,
    colframe=black,
    sharp corners=downhill,
    boxrule=1pt,
    left=5pt,
    right=5pt,
    top=5pt,
    bottom=5pt,
    boxsep=0pt,
	 % <-- add four values for each corner
  }
}
\newtcolorbox{graybox}{graybox}

%==========================================================================================



\usepackage{xcolor}
\usepackage{varwidth}
\usepackage{varwidth}
\usepackage{etoolbox}
%\usepackage{authblk}
\usepackage{nameref}
\usepackage{multicol,array}
\usepackage{tikz-cd}
\usepackage[ruled,vlined,linesnumbered]{algorithm2e}
\usepackage{comment} % enables the use of multi-line comments (\ifx \fi) 
\usepackage{import}
\usepackage{xifthen}
\usepackage{pdfpages}
\usepackage{transparent}


%\usepackage[french]{babel}
\usepackage{listings} % pour écrire du code dans un environnement
\lstset{
  basicstyle=\ttfamily,
  columns=fullflexible,
  keepspaces=true
}
\usepackage{caption}
\usepackage{float} % Pour forcer les images au bon endroit



\usepackage[T1]{fontenc}
\usepackage{csquotes}
%%%%%%%%%%%%%%%%%%%%%%%%%%%%%%%%%%%%%%%%%%%%%%%%%%%%%%%%%%%%%%%%%%%%%%%%%%%%%%%%%%%%%%%%%%%%%%%%%
%									ENSEMBLE DE COULEURS
%%%%%%%%%%%%%%%%%%%%%%%%%%%%%%%%%%%%%%%%%%%%%%%%%%%%%%%%%%%%%%%%%%%%%%%%%%%%%%%%%%%%%%%%%%%%%%%%%

\definecolor{myg}{RGB}{56, 140, 70}
\definecolor{myb}{RGB}{45, 111, 177}

\definecolor{mygbg}{RGB}{235, 253, 241}


\definecolor{myr}{RGB}{199, 68, 64}
\definecolor{mytheorembg}{HTML}{F2F2F9}
\definecolor{mytheoremfr}{HTML}{00007B}
\definecolor{mylenmabg}{HTML}{FFFAF8}
\definecolor{mylenmafr}{HTML}{983b0f}
\definecolor{mypropbg}{HTML}{f2fbfc}
\definecolor{mypropfr}{HTML}{191971}
\definecolor{myexamplebg}{HTML}{F2FBF8}
\definecolor{myexamplefr}{HTML}{88D6D1}
\definecolor{myexampleti}{HTML}{2A7F7F}
\definecolor{mydefinitbg}{HTML}{E5E5FF}
\definecolor{mydefinitfr}{HTML}{3F3FA3}
\definecolor{notesgreen}{RGB}{0,162,0}
\definecolor{myp}{RGB}{197, 92, 212}
\definecolor{mygr}{HTML}{2C3338}
\definecolor{myred}{RGB}{127,0,0}
\definecolor{myyellow}{RGB}{169,121,69}
\definecolor{myexercisebg}{HTML}{F2FBF8}
\definecolor{myexercisefg}{HTML}{88D6D1}
\definecolor{myred}{RGB}{127,0,0}
\definecolor{myyellow}{RGB}{169,121,69}

\definecolor{blue}{HTML}{008ED7}
\definecolor{mygray}{gray}{0.75}
\definecolor{lightBlue}{RGB}{235, 245, 255}
\definecolor{tcbcolred}{RGB}{255,0,0}
\definecolor{myGreen}{HTML}{009900}

% command to circle a text
\newtcbox{\entoure}[1][red]{on line,
	arc=3pt,colback=#1!10!white,colframe=#1!50!black,
	before upper={\rule[-3pt]{0pt}{10pt}},boxrule=1pt,
	boxsep=0pt,left=2pt,right=2pt,top=1pt,bottom=.5pt}
% command for the circle for the number of part entries
\newcommand\Circle[1]{\tikz[overlay,remember picture]
	\node[draw,circle, text width=18pt,line width=1pt] {#1};}

\newtcbox{\entouree}[1][red]{on line,
	arc=3pt,colback=#1!10!white,colframe=#1!50!white,
	before upper={\rule[-3pt]{0pt}{10pt}},boxrule=1pt,
	boxsep=0pt,left=2pt,right=2pt,top=1pt,bottom=.5pt}

\newcommand{\shellcmd}[1]{\\\indent\indent\texttt{\footnotesize\# #1}\\}

%=====================================================================
\patchcmd{\tableofcontents}{\contentsname}{\sffamily\contentsname}{}{}
% patching of \@part to typeset the part number inside a framed box in its own line
% and to add color
\makeatletter
\patchcmd{\@part}
  {\addcontentsline{toc}{part}{\thepart\hspace{1em}#1}}
  {\addtocontents{toc}{\protect\addvspace{20pt}}
    \addcontentsline{toc}{part}{\huge{\protect\color{myyellow}%
      \setlength\fboxrule{2pt}\protect\Circle{%
        \hfil\thepart\hfil%
      }%
    }\\[2ex]\color{myred}\sffamily#1}}{}{}

%\patchcmd{\@part}
%  {\addcontentsline{toc}{part}{\thepart\hspace{1em}#1}}
%  {\addtocontents{toc}{\protect\addvspace{20pt}}
%    \addcontentsline{toc}{part}{\huge{\protect\color{myyellow}%
%      \setlength\fboxrule{2pt}\protect\fbox{\protect\parbox[c][1em][c]{1.5em}{%
%        \hfil\thepart\hfil%
%      }}%
%    }\\[2ex]\color{myred}\sffamily#1}}{}{}
\makeatother
% this is the environment used to typeset the chapter entries in the ToC
% it is a modification of the leftbar environment of the framed package
\renewenvironment{leftbar}
  {\def\FrameCommand{\hspace{6em}%
    {\color{myyellow}\vrule width 2pt depth 6pt}\hspace{1em}}%
    \MakeFramed{\parshape 1 0cm \dimexpr\textwidth-6em\relax\FrameRestore}\vskip2pt%
  }
 {\endMakeFramed}

% using titletoc we redefine the ToC entries for parts, chapters, sections, and subsections
\titlecontents{part}
  [0em]{\centering}
  {\contentslabel}
  {}{}
\titlecontents{chapter}
  [0em]{\vspace*{2\baselineskip}}
  {\parbox{4.5em}{%
    \hfill\Huge\sffamily\bfseries\color{myred}\thecontentspage}%
   \vspace*{-2.3\baselineskip}\leftbar\textsc{\small\chaptername~\thecontentslabel}\\\sffamily}
  {}{\endleftbar}
\titlecontents{section}
  [8.4em]
  {\sffamily\contentslabel{3em}}{}{}
  {\hspace{0.5em}\nobreak\itshape\color{myred}\normalfont\contentspage}
\titlecontents{subsection}
  [8.4em]
  {\sffamily\contentslabel{3em}}{}{}  
  {\hspace{0.5em}\nobreak\itshape\color{myred}\contentspage}
%==========================================================================

%PYTHON LSTLISTING STYLE

% Define colors
\definecolor{Pgruvbox-bg}{HTML}{282828}
\definecolor{Pgruvbox-fg}{HTML}{ebdbb2}
\definecolor{Pgruvbox-red}{HTML}{fb4934}
\definecolor{Pgruvbox-green}{HTML}{b8bb26}
\definecolor{Pgruvbox-yellow}{HTML}{fabd2f}
\definecolor{Pgruvbox-blue}{HTML}{83a598}
\definecolor{Pgruvbox-purple}{HTML}{d3869b}
\definecolor{Pgruvbox-aqua}{HTML}{8ec07c}

% Define Python style
\lstdefinestyle{PythonGruvbox}{
	language=Python,
	identifierstyle=\color{lst-fg},
	basicstyle=\ttfamily\color{Pgruvbox-fg},
	keywordstyle=\color{Pgruvbox-yellow},
	keywordstyle=[2]\color{Pgruvbox-blue},
	stringstyle=\color{Pgruvbox-green},
	commentstyle=\color{Pgruvbox-aqua},
	backgroundcolor=\color{Pgruvbox-bg},
	%frame=tb,
	rulecolor=\color{Pgruvbox-fg},
	showstringspaces=false,
	keepspaces=true,
	captionpos=b,
	breaklines=true,
	tabsize=4,
	showspaces=false,
	numbers=left,
	numbersep=5pt,
	numberstyle=\tiny\color{gray},
	showtabs=false,
	columns=fullflexible,
	morekeywords={True,False,None},
	morekeywords=[2]{and,as,assert,break,class,continue,def,del,elif,else,except,exec,finally,for,from,global,if,import,in,is,lambda,nonlocal,not,or,pass,print,raise,return,try,while,with,yield},
	morecomment=[s]{"""}{"""},
	morecomment=[s]{'''}{'''},
	morecomment=[l]{\#},
	morestring=[b]",
	morestring=[b]',
	literate=
	{0}{{\textcolor{Pgruvbox-purple}{0}}}{1}
	{1}{{\textcolor{Pgruvbox-purple}{1}}}{1}
	{2}{{\textcolor{Pgruvbox-purple}{2}}}{1}
	{3}{{\textcolor{Pgruvbox-purple}{3}}}{1}
	{4}{{\textcolor{Pgruvbox-purple}{4}}}{1}
	{5}{{\textcolor{Pgruvbox-purple}{5}}}{1}
	{6}{{\textcolor{Pgruvbox-purple}{6}}}{1}
	{7}{{\textcolor{Pgruvbox-purple}{7}}}{1}
	{8}{{\textcolor{Pgruvbox-purple}{8}}}{1}
	{9}{{\textcolor{Pgruvbox-purple}{9}}}{1}
}
%====================================================================
% 
%====================================================================

% JAVA LSTLISTING STYLE IN Gruvbox Colorscheme
\definecolor{gruvbox-bg}{rgb}{0.282, 0.247, 0.204}
\definecolor{gruvbox-fg1}{rgb}{0.949, 0.898, 0.776}
\definecolor{gruvbox-fg2}{rgb}{0.871, 0.804, 0.671}
\definecolor{gruvbox-red}{rgb}{0.788, 0.255, 0.259}
\definecolor{gruvbox-green}{rgb}{0.518, 0.604, 0.239}
\definecolor{gruvbox-yellow}{rgb}{0.914, 0.808, 0.427}
\definecolor{gruvbox-blue}{rgb}{0.353, 0.510, 0.784}
\definecolor{gruvbox-purple}{rgb}{0.576, 0.412, 0.659}
\definecolor{gruvbox-aqua}{rgb}{0.459, 0.631, 0.737}
\definecolor{gruvbox-gray}{rgb}{0.518, 0.494, 0.471}

\definecolor{lst-bg}{RGB}{45, 45, 45}
\definecolor{lst-fg}{RGB}{220, 220, 204}
\definecolor{lst-keyword}{RGB}{215, 186, 125}
\definecolor{lst-comment}{RGB}{117, 113, 94}
\definecolor{lst-string}{RGB}{163, 190, 140}
\definecolor{lst-number}{RGB}{181, 206, 168}
\definecolor{lst-type}{RGB}{218, 142, 130}


\lstdefinestyle{JavaGruvbox}{
	language=Java,
	basicstyle=\ttfamily\color{lst-fg},
	keywordstyle=\color{lst-keyword},
	keywordstyle=[2]\color{lst-type},
	commentstyle=\itshape\color{lst-comment},
	stringstyle=\color{lst-string},
	numberstyle=\color{lst-number},
	backgroundcolor=\color{lst-bg},
	%frame=tb,
	rulecolor=\color{gruvbox-aqua},
	showstringspaces=false,
	keepspaces=true,
	captionpos=b,
	breaklines=true,
	tabsize=4,
	showspaces=false,
	showtabs=false,
	columns=fullflexible,
	morekeywords={var},
	morekeywords=[2]{boolean, byte, char, double, float, int, long, short, void},
	morecomment=[s]{/}{/},
	morecomment=[l]{//},
	morestring=[b]",
	morestring=[b]',
	numbers=left,
	numbersep=5pt,
	numberstyle=\tiny\color{gray},
}



%====================================================================
% 
%====================================================================


% Define Dracula color scheme for Java
\definecolor{draculawhite-background}{RGB}{237, 239, 252}
\definecolor{draculawhite-comment}{RGB}{98, 114, 164}
\definecolor{draculawhite-keyword}{RGB}{189, 147, 249}
\definecolor{draculawhite-string}{RGB}{152, 195, 121}
\definecolor{draculawhite-number}{RGB}{249, 189, 89}
\definecolor{draculawhite-operator}{RGB}{248, 248, 242}

% Define JavaDraculaWhite lstlisting environment
\lstdefinestyle{JavaDraculaWhite}{
    language=Java,
    backgroundcolor=\color{draculawhite-background},
    commentstyle=\itshape\color{draculawhite-comment},
    keywordstyle=\color{draculawhite-keyword},
    stringstyle=\color{draculawhite-string},
    basicstyle=\ttfamily\small\color{black},
    identifierstyle=\color{black},
    keywordstyle=\color{draculawhite-keyword}\bfseries,
    morecomment=[s][\color{draculawhite-comment}]{/**}{*/},
    showstringspaces=false,
    showspaces=false,
    breaklines=true,
    frame=single,
    rulecolor=\color{draculawhite-operator},
    tabsize=4,  
	numbers=left,
	numbersep=4pt,
	numberstyle=\ttfamily\tiny\color{gray}
}
%====================================================================
% 
%====================================================================
% Define PythonDraculaWhite lstlisting environment 
\definecolor{draculawhite-bg}{HTML}{FAFAFA}
\definecolor{draculawhite-fg}{HTML}{282A36}
\definecolor{pdraculawhite-keyword}{HTML}{BD93F9}

\definecolor{pdraculawhite-comment}{HTML}{6272A4}
\definecolor{draculawhite-number}{HTML}{FF79C6}


\lstdefinestyle{PythonDraculaWhite}{
    language=Python,
    basicstyle=\ttfamily\small\color{draculawhite-fg},
    backgroundcolor=\color{draculawhite-background},
    keywordstyle=\color{orange}\bfseries,
    stringstyle=\color{draculawhite-string},
    commentstyle=\color{pdraculawhite-comment}\itshape,
    numberstyle=\color{draculawhite-number},
    showstringspaces=false,
	showspaces=false,
    breaklines=true,
	frame=single,
	rulecolor=\color{draculawhite-operator}, 
    tabsize=4,
    morekeywords={as,with,1,2,3,4, 5,6,7,8,9,True,False,@media},
    %escapeinside={(*@}{@*)},
    numbers=left,
    numbersep=5pt,
    %xleftmargin=15pt,
    %framexleftmargin=15pt,
    %framexrightmargin=0pt,
    %framexbottommargin=0pt,
    %framextopmargin=0pt,
    %rulecolor=\color{draculawhite-fg},
    %frame=tb,
    %aboveskip=0pt,
    %belowskip=0pt,
    %captionpos=b,
	numberstyle=\ttfamily\tiny\color{gray} 
}
%====================================================================
% 
%====================================================================

% Define colors for HTML langage
\definecolor{html-orange}{HTML}{FF5733}
\definecolor{html-yellow}{HTML}{F0E130}
\definecolor{html-green}{HTML}{50FA7B}
\definecolor{html-blue}{HTML}{5AFBFF}
\definecolor{html-purple}{HTML}{BD93F9}
\definecolor{html-pink}{HTML}{FF80BF}
\definecolor{html-gray}{HTML}{6272A4}
\definecolor{html-white}{HTML}{F8F8F2}

% Defines a new HTML5 langage that extend on the html langange
\lstdefinestyle{HTMLDraculaWhite}{
  language=HTML,
  backgroundcolor=\color{html-white},
  basicstyle=\ttfamily\color{html-gray},
  keywordstyle=\color{html-blue},
  stringstyle=\color{html-orange},
  commentstyle=\color{html-green},
  tagstyle=\color{html-yellow},
  moredelim=[s][\color{html-pink}]{<!--}{-->},
  moredelim=[s][\color{html-purple}]{\{}{\}},
  showstringspaces=false,
  tabsize=2,
  breaklines=true,
  columns=fullflexible,
  %frame=single,
  framexleftmargin=5mm,
  xleftmargin=10mm,
  numbers=left,
  numberstyle=\tiny\color{html-gray},
  escapeinside={<@}{@>},
}

%====================================================================
% 
%====================================================================
% Define the colors needed for the HTMLDraculaDark environment
\definecolor{htmltag}{HTML}{ff79c6}
\definecolor{htmlattr}{HTML}{f1fa8c}
\definecolor{htmlvalue}{HTML}{bd93f9}
\definecolor{htmlcomment}{HTML}{6272a4}
%\definecolor{htmltext}{HTML}{f8f8f2}
\definecolor{htmltext}{HTML}{401E31}
\definecolor{htmlbackground}{HTML}{282a36}
\definecolor{comphtmlbackground}{HTML}{8093FF}
%\definecolor{htmlbackground}{HTML}{4D5169}

% Define the HTMLDraculaDark environment
\lstdefinestyle{HTMLDraculaDark}{
    basicstyle=\bfseries\ttfamily\color{htmltext},
    commentstyle=\itshape\color{htmlcomment},
    keywordstyle=\bfseries\color{htmltag},
    stringstyle=\color{htmlvalue},
    emph={DOCTYPE,html,head,body,div,span,a,script},
    emphstyle={\color{htmltag}\bfseries},
    sensitive=true,
    showstringspaces=false,
    backgroundcolor=\color{white},
    %frame=tb,
    language=HTML,
    tabsize=4,
    breaklines=true,
    breakatwhitespace=true,
    numbers=left,
    numbersep=10pt,
    numberstyle=\small\bfseries\ttfamily\color{htmlcomment},
    escapeinside={<@}{@>},
	rulecolor=\color{htmlbackground},
	xleftmargin=20pt,
	% Add a vertical line for opening and closing tags
    %frame=lines,
    framesep=2pt,
    framexleftmargin=4pt,
    % Add a vertical line for closing tags that go through multiple lines
    breaklines=true,
    postbreak=\mbox{\textcolor{gray}{$\hookrightarrow$}\space},
    showlines=true,
	% Add a rule to apply commentstyle to keywords inside comments
    moredelim=[s][\itshape\color{htmlcomment}]{<!--}{-->},
    morekeywords={id,class,type,name,value,placeholder,checked,src,href,alt}
}
%====================================================================
% 
%====================================================================
\lstdefinelanguage{CSS}{
  keywords={color,background,font-weight,style,family,size,text-decoration,font-family,text-align,font-size,font-weight,border, padding, margin,},
  %keywordstyle=\color{CSSSelector},
  %keywords={[2]font-size},
  %keywordstyle={[2]\color{CSSProperty}},
  moredelim=*[s][\color{CSSValue}]{:}{;},
  moredelim=*[s][\color{orange}]{\ }{,},
  alsoletter=-,
  sensitive=true,
  comment=[l]{//},
  morecomment=[s]{/*}{*/},
  commentstyle=\color{CSSComment},
  stringstyle=\color{CSSValue},
  morestring=[b]',
  morestring=[b]",
}



\definecolor{CSSComment}{HTML}{6272A4}
\definecolor{CSSSelector}{HTML}{50FA7B}
\definecolor{CSSProperty}{HTML}{FF79C6}
\definecolor{CSSValue}{HTML}{BD93F9}

\lstdefinestyle{CSSDraculaDark}{
  language=css,
  basicstyle=\bfseries\ttfamily,
  commentstyle=\itshape\color{CSSComment},
  keywordstyle=\bfseries\color{CSSProperty},
  %keywordstyle={[2]\color{blue}},
  %keywordstyle={[3]\color{blue}},
  %keywordstyle={[4]\color{blue}},
  stringstyle=\color{CSSValue},
  morecomment=[s]{/*}{*/},
  morekeywords={color,background},
  morekeywords={[2]font-size},
  frame=tb,
  framesep=4pt,
  framerule=0pt,
  showspaces=false,
  showstringspaces=false,
  breaklines=true,
  breakatwhitespace=true,
  tabsize=4,
  numbers=left,
  numberstyle=\small\bfseries\ttfamily\color{htmlcomment} ,
  numbersep=10pt,
  xleftmargin=20pt,
  aboveskip=5pt,
  belowskip=5pt,
}


%====================================================================
% 
%====================================================================
% Crée un environnement "Theorem" numéroté en fonction du document
\tcbuselibrary{theorems,skins,hooks} 
\newtcbtheorem{Theorem}{Théorème}
{%
	enhanced,
	breakable,
	colback = mytheorembg,
	frame hidden,
	boxrule = 0sp,
	borderline west = {2pt}{0pt}{mytheoremfr},
	sharp corners,
	detach title,
	before upper = \tcbtitle\par\smallskip,
	coltitle = mytheoremfr,
	fonttitle = \bfseries\fontfamily{lmss}\selectfont,
	description font = \mdseries\fontfamily{lmss}\selectfont,
	separator sign none,
	segmentation style={solid, mytheoremfr},
}
{thm}

% Crée un environnement "Preuve" numéroté en fonction du document
\tcbuselibrary{theorems,skins,hooks}
\newtcbtheorem{Preuve}{Preuve.}
{
	enhanced,
	breakable,
	colback=white,
	frame hidden,
	boxrule = 0sp,
	borderline west = {2pt}{0pt}{mytheoremfr},
	sharp corners,
	detach title,
	before upper = \tcbtitle\par\smallskip,
	coltitle = mytheoremfr,
	description font=\fontfamily{lmss}\selectfont,
	fonttitle=\fontfamily{lmss}\selectfont\bfseries,
	separator sign none,
	segmentation style={solid, mytheoremfr},
}
{th}


% Crée un environnement "Preuve" numéroté en fonction du document
\tcbuselibrary{theorems,skins,hooks}
\newtcbtheorem{Explication}{Explication}
{
	enhanced,
	breakable,
	colback=white,
	frame hidden,
	boxrule = 0sp,
	borderline west = {2pt}{0pt}{mytheoremfr},
	sharp corners,
	detach title,
	before upper = \tcbtitle\par\smallskip,
	coltitle = mytheoremfr,
	description font=\fontfamily{lmss}\selectfont,
	fonttitle=\fontfamily{lmss}\selectfont\bfseries,
	separator sign none,
	segmentation style={solid, mytheoremfr},
}
{th}




% Crée un environnement "Example" numéroté en fonction du document
\tcbuselibrary{theorems,skins,hooks}
\newtcbtheorem{Example}{Exemple.}
{
	enhanced,
	breakable,
	colback=lightBlue,
	frame hidden,
	boxrule = 0sp,
	borderline west = {2pt}{0pt}{myb},
	sharp corners,
	detach title,
	before upper = \tcbtitle\par\smallskip,
	coltitle = myb,
	description font=\fontfamily{lmss}\selectfont,
	fonttitle=\fontfamily{lmss}\selectfont\bfseries,
	separator sign none,
	segmentation style={solid, mytheoremfr},
}
{th}



% Crée un environnement "EExample" numéroté en fonction du document
\tcbuselibrary{theorems,skins,hooks}
\newtcbtheorem{EExample}{Exemple.}
{
	enhanced,
	breakable,
	colback=white,
	frame hidden,
	boxrule = 0sp,
	borderline west = {2pt}{0pt}{myb},
	sharp corners,
	detach title,
	before upper = \tcbtitle\par\smallskip,
	coltitle = myb,
	description font=\mdseries\fontfamily{lmss}\selectfont,
	fonttitle=\fontfamily{lmss}\selectfont\bfseries,
	separator sign none,
	segmentation style={solid, mytheoremfr},
}
{th}



% Crée un environnement "ExampleDdHTML" numéroté en fonction du document
\tcbuselibrary{theorems,skins,hooks}
\newtcbtheorem{ExampleDdHTML}{Exemple.}
{
	enhanced,
	breakable,
	colback=white,
	frame hidden,
	boxrule = 0sp,
	borderline west = {2pt}{0pt}{htmlbackground},
	sharp corners,
	detach title,
	before upper = \tcbtitle\par\smallskip,
	coltitle = htmlbackground,
	description font=\mdseries\fontfamily{lmss}\selectfont,
	fonttitle=\fontfamily{lmss}\selectfont\bfseries,
	separator sign none,
	segmentation style={solid, mytheoremfr},
}
{th}




% Crée un environnement "Lemme" numéroté en fonction du document
\tcbuselibrary{theorems,skins,hooks}
\newtcbtheorem{Lemme}{Lemme}
{
	enhanced,
	breakable,
	colback=mylenmabg,
	frame hidden,
	boxrule = 0sp,
	borderline west = {2pt}{0pt}{mylenmafr},
	sharp corners,
	detach title,
	before upper = \tcbtitle\par\smallskip,
	coltitle = mylenmafr,
	description font=\mdseries\fontfamily{lmss}\selectfont,
	fonttitle=\fontfamily{lmss}\selectfont\bfseries,
	separator sign none,
	segmentation style={solid, mytheoremfr},
}
{th}


\tcbuselibrary{theorems,skins,hooks}
\newtcbtheorem{PreuveL}{Preuve.}
{
	enhanced,
	breakable,
	colback=white,
	frame hidden,
	boxrule = 0sp,
	borderline west = {2pt}{0pt}{mylenmafr},
	sharp corners,
	detach title,
	before upper = \tcbtitle\par\smallskip,
	coltitle = mylenmafr,
	description font=\fontfamily{lmss}\selectfont,
	fonttitle=\fontfamily{lmss}\selectfont\bfseries,
	separator sign none,
	segmentation style={solid, mytheoremfr},
}
{th}


\newtcbtheorem{Remarque}{Remarque.}
{
	enhanced,
	breakable,
	colback=white,
	frame hidden,
	boxrule = 0sp,
	borderline west = {2pt}{0pt}{myb},
	sharp corners,
	detach title,
	before upper = \tcbtitle\par\smallskip,
	coltitle = myb,
	description font=\mdseries\fontfamily{lmss}\selectfont,
	fonttitle=\fontfamily{lmss}\selectfont\bfseries,
	separator sign none,
	segmentation style={solid, mytheoremfr},
}
{th}


\newtcbtheorem{DefG}{Définition}
{
	enhanced,
	breakable,
	colback=mygbg,
	frame hidden,
	boxrule = 0sp,
	borderline west = {2pt}{0pt}{myg},
	sharp corners,
	detach title,
	before upper = \tcbtitle\par\smallskip,
	coltitle = myg,
	description font=\mdseries\fontfamily{lmss}\selectfont,
	fonttitle=\fontfamily{lmss}\selectfont\bfseries,
	separator sign none,
	segmentation style={solid, mytheoremfr},
}
{th}



% Crée une boîte ayant la même couleur que l'environnement theorem.
\tcbuselibrary{theorems,skins,hooks}
\newtcolorbox{Theoremcon}
{%
	enhanced
	,breakable
	,colback = mytheorembg
	,frame hidden
	,boxrule = 0sp
	,borderline west = {2pt}{0pt}{mytheoremfr}
	,sharp corners
	,description font = \mdseries
	,separator sign none
}

% Crée un environnement "Definition" numéroté en fonction de la section
\newtcbtheorem[number within=chapter]{Definition}{Définition}{enhanced,
	before skip=2mm,after skip=2mm, colback=red!5,colframe=red!80!black,boxrule=0.5mm,
	attach boxed title to top left={xshift=1cm,yshift*=1mm-\tcboxedtitleheight}, varwidth boxed title*=-3cm,
	boxed title style={frame code={
			\path[fill=tcbcolback!10!red]
			([yshift=-1mm,xshift=-1mm]frame.north west)
			arc[start angle=0,end angle=180,radius=1mm]
			([yshift=-1mm,xshift=1mm]frame.north east)
			arc[start angle=180,end angle=0,radius=1mm];
			\path[left color=tcbcolback!10!myred,right color=tcbcolback!10!myred,
			middle color=tcbcolback!60!myred]
			([xshift=-2mm]frame.north west) -- ([xshift=2mm]frame.north east)
			[rounded corners=1mm]-- ([xshift=1mm,yshift=-1mm]frame.north east)
			-- (frame.south east) -- (frame.south west)
			-- ([xshift=-1mm,yshift=-1mm]frame.north west)
			[sharp corners]-- cycle;
		},interior engine=empty,
	},
	fonttitle=\bfseries,
	title={#2},#1}{def}

% Crée un environnement "definition" numéroté en fonction du Chapitre
\newtcbtheorem[number within=section]{definition}{Définition}{enhanced,
	before skip=2mm,after skip=2mm, colback=red!5,colframe=red!80!black,boxrule=0.5mm,
	attach boxed title to top left={xshift=1cm,yshift*=1mm-\tcboxedtitleheight}, varwidth boxed title*=-3cm,
	boxed title style={frame code={
			\path[fill=tcbcolback]
			([yshift=-1mm,xshift=-1mm]frame.north west)
			arc[start angle=0,end angle=180,radius=1mm]
			([yshift=-1mm,xshift=1mm]frame.north east)
			arc[start angle=180,end angle=0,radius=1mm];
			\path[left color=tcbcolback!60!black,right color=tcbcolback!60!black,
			middle color=tcbcolback!80!black]
			([xshift=-2mm]frame.north west) -- ([xshift=2mm]frame.north east)
			[rounded corners=1mm]-- ([xshift=1mm,yshift=-1mm]frame.north east)
			-- (frame.south east) -- (frame.south west)
			-- ([xshift=-1mm,yshift=-1mm]frame.north west)
			[sharp corners]-- cycle;
		},interior engine=empty,
	},
	fonttitle=\bfseries,
	title={#2},#1}{def}

\usetikzlibrary{arrows,calc,shadows.blur}
\tcbuselibrary{skins}
\newtcolorbox{note}[1][]{%
	enhanced jigsaw,
	colback=gray!20!white,%
	colframe=gray!80!black,
	size=small,
	boxrule=1pt,
	title=\textbf{Note : },
	halign title=flush center,
	coltitle=black,
	breakable,
	drop shadow=black!50!white,
	attach boxed title to top left={xshift=1cm,yshift=-\tcboxedtitleheight/2,yshifttext=-\tcboxedtitleheight/2},
	minipage boxed title=1.5cm,
	boxed title style={%
		colback=white,
		size=fbox,
		boxrule=1pt,
		boxsep=2pt,
		underlay={%
			\coordinate (dotA) at ($(interior.west) + (-0.5pt,0)$);
			\coordinate (dotB) at ($(interior.east) + (0.5pt,0)$);
			\begin{scope}
				\clip (interior.north west) rectangle ([xshift=3ex]interior.east);
				\filldraw [white, blur shadow={shadow opacity=60, shadow yshift=-.75ex}, rounded corners=2pt] (interior.north west) rectangle (interior.south east);
			\end{scope}
			\begin{scope}[gray!80!black]
				\fill (dotA) circle (2pt);
				\fill (dotB) circle (2pt);
			\end{scope}
		},
	},
	#1,
}


% Crée un environnement "qstion" 
\newtcbtheorem{qstion}{Question}{enhanced,
	breakable,
	colback=white,
	colframe=mygr,
	attach boxed title to top left={yshift*=-\tcboxedtitleheight},
	fonttitle=\bfseries,
	title={#2},
	boxed title size=title,
	boxed title style={%
		sharp corners,
		rounded corners=northwest,
		colback=tcbcolframe,
		boxrule=0pt,
	},
}{def}


% Pour créer un environnement "Liste" 

\tcbuselibrary{theorems,skins,hooks}
\newtcbtheorem[number within=section]{Liste}{Liste}
{%
	enhanced
	,breakable
	,colback = myp!10
	,frame hidden
	,boxrule = 0sp
	,borderline west = {2pt}{0pt}{myp!85!black}
	,sharp corners
	,detach title
	,before upper = \tcbtitle\par\smallskip
	,coltitle = myp!85!black
	,fonttitle = \bfseries\sffamily
	,description font = \mdseries
	,separator sign none
	,segmentation style={solid, myp!85!black}
}
{th}


\tcbuselibrary{theorems,skins,hooks}
\newtcbtheorem{Syntaxe}{Syntaxe.}
{%
	enhanced
	,breakable
	,colback = myp!10
	,frame hidden
	,boxrule = 0sp
	,borderline west = {2pt}{0pt}{myp!85!black}
	,sharp corners
	,detach title
	,before upper = \tcbtitle\par\smallskip
	,coltitle = myp!85!black
	,fonttitle = \bfseries\fontfamily{lmss}\selectfont 
	,description font = \mdseries\fontfamily{lmss}\selectfont 
	,separator sign none
	,segmentation style={solid, myp!85!black}
}
{th}



% Crée un environnement "Concept" numéroté en fonction du document
\tcbuselibrary{theorems,skins,hooks}
\newtcbtheorem{Concept}{Concept.}
{
	enhanced,
	breakable,
	colback=mylenmabg,
	frame hidden,
	boxrule = 0sp,
	borderline west = {2pt}{0pt}{mylenmafr},
	sharp corners,
	detach title,
	before upper = \tcbtitle\par\smallskip,
	coltitle = mylenmafr,
	description font=\mdseries\fontfamily{lmss}\selectfont,
	fonttitle=\fontfamily{lmss}\selectfont\bfseries,
	separator sign none,
	segmentation style={solid, mytheoremfr},
}
{th}


% Crée un environnement "codeEx" numéroté en fonction du document
\tcbuselibrary{theorems,skins,hooks}
\newtcbtheorem{codeEx}{Exemple.}
{
	enhanced,
	breakable,
	colback=white,
	frame hidden,
	boxrule = 0sp,
	borderline west = {2pt}{0pt}{gruvbox-bg},
	sharp corners,
	detach title,
	before upper = \tcbtitle\par\smallskip,
	coltitle = gruvbox-bg,
	description font=\mdseries\fontfamily{lmss}\selectfont,
	fonttitle=\fontfamily{lmss}\selectfont\bfseries,
	separator sign none,
	segmentation style={solid, mytheoremfr},
}
{th}


% Crée un environnement "codeEx" numéroté en fonction du document
\tcbuselibrary{theorems,skins,hooks}
\newtcbtheorem{codeRem}{Remarque.}
{
	enhanced,
	breakable,
	colback=white,
	frame hidden,
	boxrule = 0sp,
	borderline west = {2pt}{0pt}{gruvbox-bg},
	sharp corners,
	detach title,
	before upper = \tcbtitle\par\smallskip,
	coltitle = gruvbox-bg,
	description font=\mdseries\fontfamily{lmss}\selectfont,
	fonttitle=\fontfamily{lmss}\selectfont\bfseries,
	separator sign none,
	segmentation style={solid, mytheoremfr},
}
{th}


\tcbuselibrary{theorems,skins,hooks}










\title{\Huge{Harvard CS30W}\\HTML and CSS}
\author{\huge{Franz Girardin}}
\date{1 Mai 2023}
\usepackage{framed}

%====================================================================

%====================================================================
\begin{document}
\maketitle

\newpage
\pdfbookmark[section]{\contentsname}{toc}
\tableofcontents
\pagebreak
%====================================================================
% 
%====================================================================
\chapter{HTML}
\begin{DefG}{HTML \textit{Hypertext Markup Langage}}{}
Langage de balisage qu'on utilise pour décrire la structure d'un document
destiné à afficher une page web.
\end{DefG}


\begin{DefG}{Déclaration doctype}{}
Formule qu'on utilise pour préciser la version de HTML utilisé dans la page qui suit : \texttt{\footnotesize{<!Doctype html>}}
\end{DefG}


\begin{Remarque*}{}{}
Lorsqu'on précise \texttt{\footnotesize{html}} après !doctype ce signale au navigateur qu'on utilise le 
standard HTML5 pour interpréter le contenu du document HTML. 
\end{Remarque*}


\begin{Syntaxe*}{En-tête \texttt{<head>}}{}
Cette section contient des informations d'ordre générale concernant la page. Notamment, on y trouve le titre de la page : \texttt{\footnotesize{<title> </title>}} 
\end{Syntaxe*}


\begin{figure}[H]
	\centering
	\caption{Structure générale d'un document HTML}
	\includegraphics[width=0.70\textwidth]{helloworld2} 
\end{figure} 



\begin{DefG}{DOM \textit{Document Object Model}}{}
Structure qui permet de représenter les différents éléments d'une page web et la relation hiérarchique entre
chacun 
\end{DefG}



\begin{figure}[H]
	\centering
	\caption{DOM d'un document HTML}
	\includegraphics[width=0.35\textwidth]{DOMhtml} 
\end{figure}


\section{Listes HTML}
\noindent Il est possible de créer des listes en faisant appel aux balises \entouree[gray]{\texttt{\footnotesize{<ol> </ol>}}} et \entouree[gray]{\texttt{\footnotesize{<ul> </ul>}}} qui créent des listes ordonnées et non ordonées, respectivement. Chaque élément d'une liste est introduit par la balise \entouree[gray]{\texttt{\footnotesize{<li></li>}}}. 


\definecolor{shadecolor}{HTML}{282a36} 

\begin{ExampleDdHTML*}{}{}
  \begin{lstlisting}[style=HTMLDraculaDark]
  <!DOCTYPE html>		<!-- Debut de la balise html contant la page web entiere -->
  <html lang="en">
	<head>
		<title>Hello!</title>
	</head>
	<body>
		An ordered List:
		<ol>
			<li>First Item</li>
			<li>Second Item</li>
			<li>Third Item</li>
		</ol>
	</body>
  </html>
  \end{lstlisting}
\end{ExampleDdHTML*}



\section{Images HTML}
\noindent Il est possible d'inclure une image grâce à la balise \entouree[gray]{\texttt{\footnotesize{<img src="..." alt="...">}}}. L'attribut \entouree[gray]{\texttt{\footnotesize{src}}} prend comme argument le lien de l'image qu'on désire inclure. L'attribut \entouree[gray]{\texttt{\footnotesize{alt}}} prend comme argument un texte qui sera affiché
si le navigateur \textit{n'arrive pas à afficher l'image}. 

\begin{ExampleDdHTML*}{}{}
  \begin{lstlisting}[style=HTMLDraculaDark]
  <!DOCTYPE html>
  <html lang="en">
	<head>
		<title>image</title>
	</head>
	<body>
		<!-- Fournit le lien de l'image cat.jgp --> 
		<img src="cat.jpg" alt="imageDeChat" width="300"> 
	</body>
  </html>
  \end{lstlisting}
\end{ExampleDdHTML*}

\section{Liens HTML}
\noindent Il est possible de partager des liens grâce à la balise \entouree[gray]{\texttt{\footnotesize{<a href="..."}></a>}}.
La balise contient un text qui s'affichera comme un lien cliquable. 


\begin{ExampleDdHTML*}{}{}
  \begin{lstlisting}[style=HTMLDraculaDark]
  <!DOCTYPE html>
  <html lang="en">
	<head>
		<title>lien</title>
	</head>
	<body>
		<!-- Fournit un lien vers la page google.com-->
		<a href="https://google.com">Click Here</a>
	</body>
  </html>
  \end{lstlisting}
\end{ExampleDdHTML*}



\section{Tableau HTML}
\noindent Il est possible de générer un tableau grâce à la balise \entouree[gray]{\texttt{\footnotesize{<table></table>}}}. Cette balise est un contenant pour la balise \entouree[gray]{\texttt{\footnotesize{<thead></thead>}}} qui 
permet de générer une \textit{en-tête} pour le tableau. Chaque élément de l'entête est introduit par
\entouree[gray]{\texttt{\footnotesize{<th></th>}}}. On peut ensuite introduire les éléments du tableau grâce au 
contenant \entouree[gray]{\texttt{\footnotesize{<tbody></tbody>}}}. Dans le corps du tableau, chaque rangée est
introduite par \entouree[gray]{\texttt{\footnotesize{<tr></tr>}}}. On ajoute les éléments grâce à \entouree[gray]{\texttt{\footnotesize{<td></td>}}}
\begin{ExampleDdHTML*}{}{}
  \begin{lstlisting}[style=HTMLDraculaDark] 
  <!DOCTYPE html>
  <html lang="en">
	<head>
		<title>Tableau</title>
	</head>
	<body>
		<table>
			<tr>
				<thead>
					<th>Ocean</th>
					<th>Average Depth></th>
					<th>Maximum Depth></th>
				</thead>
			</tr>

			<tbody>
				<tr>
					<td>Pacific Ocean</td>
					<td>4,280 m</td>
					<td>10,911 m</td>
				</tr>

				<tr>
					<td>Atlantic Ocean</td>
					<td>3,4646 m</td>
					<td>8,486 m</td>
				<tr>
				
			</tbody>

		</table>
	</body>
	</html>
  \end{lstlisting}
\end{ExampleDdHTML*}

\section{Formulaire HTML}
\noindent Il est possible de créer un formulaire qui donne l'oportunité à l'usager d'entrée des données. 
La balise \entouree[gray]{\texttt{\footnotesize{<form></form>}}} introduit ce formulaire. L'entrée de données
par l'utilisateur est rendu possible par l'usage de \entouree[gray]{\texttt{\footnotesize{<input type="text"}}}.
Le \entouree[gray]{\texttt{\footnotesize{placeholder}}} est un indice visuel qui permet \ l'utilisateur de 
comprendre la donnée qui lui est demandée. 


\begin{ExampleDdHTML*}{}{}
  \begin{lstlisting}[style=HTMLDraculaDark]  
<!DOCTYPE html>
<html lang="en">
<head>
    <title>Forms</title>
</head>
<body>
    <form>
        <input type="text" placeholder="First Name" name="first">
        <input type="password" placeholder="Password" name="password">
        <div>
            Favorite Color:
            <input name="color" type="radio" value="blue"> Blue
            <input name="color" type="radio" value="green"> Green
            <input name="color" type="radio" value="yellow"> Yellow
            <input name="color" type="radio" value="red"> Red

        </div>
        <input type="submit">
    </form>
</body>
</html>	 
  \end{lstlisting}
\end{ExampleDdHTML*}
%====================================================================
% 
%====================================================================

\chapter{CSS}
\noindent Il est possible de modifier l'apparence ou \textit{le style} des éléments d'un document HTML grâce au langage CSS. L'une des façons de spécifier un style est d'utiliser l'attribut \entouree[gray]{\texttt{\footnotesize{style}}} d'une balise. Chaque style est un couple de \entouree[gray]{\texttt{\footnotesize{clé: valeur;}}}. 



\begin{ExampleDdHTML*}{}{}
  \begin{lstlisting}[style=HTMLDraculaDark]  
  <!DOCTYPE html>
  <html lang="en">
	<head>
		<title>Hello!</title>
	</head>
	<body>
		<h1 style="color: blue; text-align: center;"> Welcome to my home page</h1>
		Hello, world!
	</body>
	</html>
  \end{lstlisting}
\end{ExampleDdHTML*}
\noindent Il est également possible de modifier le style en utilisant des balises 
\entouree[gray]{\texttt{\footnotesize{<style></style>}}} dans une autre section du document


\begin{ExampleDdHTML*}{}{}
  \begin{lstlisting}[style=HTMLDraculaDark]  
  <!DOCTYPE html>
  <html lang="en">
	<head>
		<title>Hello!</title>
		<style>
			h1	{color: blue;
				 text-align: center;
			}
		</style>
	</head>
	<body>
		<h1> Welcome to my home page</h1>
		Hello, world!
	</body>
	</html>
  \end{lstlisting}
\end{ExampleDdHTML*}
La troisième alternative est d'utiliser une feuille de style CSS et d'en fournir le lien dans le document HTML

\begin{ExampleDdHTML*}{}{}
  \begin{minipage}{.5\textwidth}
  \begin{lstlisting}[style=HTMLDraculaDark]  
  <!-- document1.html-->
  <!DOCTYPE html>
  <html lang="en">
	<head>
		<title>Hello!</title>
		<link rel="stylesheet" href="style.css">
	</head>
	<body>
		<h1>Welcome to my home page</h1>
		<h1>This is another heading</h1>
		Hello, world!
	</body>
	</html>
  \end{lstlisting}
  \end{minipage}

   \begin{minipage}{.5\textwidth}
	\begin{lstlisting}[style=HTMLDraculaDark]  
	<!-- document1.css-->
	h1    {color: blue;
		   text-align: center;
		   }
	\end{lstlisting}
\end{minipage}


\end{ExampleDdHTML*}
%====================================================================
% 
%====================================================================


\chapter{Conception Réactive}
\section{Concept de réactivité}
\begin{DefG*}{Responsive Design}
	Un design réactif permet l'affichage optimale d'un page web, indépendamment de l'outil utilisé
	pour la consulter. Le design est réactif lorsqu'il s'adapte au \textit{viewport}—la portion visible de la page. 

\end{DefG*}
\begin{note}
Par défaut, les téléphones intelligents réduisent l'espace que prennent chaque éléments, pour présenter 
l'entièreté de la page telle qu'elle serait visible sur un écran d'ordinateur. 
\end{note}



\begin{Syntaxe*}{Modification du \textit{viewport}}{}
La commande \texttt{\footnotesize{<meta name="viewport" content="width=device-widht, initial-scale=1.0">}} permet d'ajuster les dimensions du \textit{viewport} à la largeur de l'écran. 
\end{Syntaxe*}


\section{Media Queries}
\noindent Ce sont des commandes qui permettent de contrôler l'apparence de la page en manipulant des paramètres de 
taille et de présentation. Dans l'exemple suivant, on modifie l'apparence de la page en fonction de la taille de 
l'écran de l'utilisateur. 


\begin{ExampleDdHTML*}{Changement de la couleur en fonction de la taille de l'écran}{}
\begin{lstlisting}[style=HTMLDraculaDark]  
  <!DOCTYPE html>
  <html lang="en">
	<head>
		<title>Changement de couleur</title>

		
		<style>
		<!--Ce query change la couleur de fond si la largeur de l'ecran est >= 600px-->
	
			@media (min-width: 600px) {
			body {
				background-color: red;
				}
			}
		<!-- Ce query change la couleur de fond si la largeur de l'ecran est <= 599 px-->
		@media (max-widthL 599px) {
		body {
			background-color: blue;
			}
		}
		</style>
	</head>
	<body>
		<h1>Welcome to my web page !</h1>
	</body>
  </html>
\end{lstlisting}
\end{ExampleDdHTML*}

\begin{Remarque*}{Créer un \textit{Media Query}}{}
Chaque media query est introduit par \texttt{\footnotesize{@media}} suivit d'une expression entre parenthèses. Dans
les parenthèses, on spécifie pour quel genre de media on veut appliquer le \textit{query}. Ce qui suit l'expression
entre parenthèse est une expression entre crochets qui spécifie le \textit{query}. 
\end{Remarque*}





\section{Flexbox}
\begin{Syntaxe*}{Flexbox}{}
L'utilisation de \textit{flexbox} peut être pertinente lorsqu'on désire présenter plusieurs éléments 
sur une page et que ces éléments pourraient sortir du champ de la page si on ne conçoit pas un design réactif. 
\end{Syntaxe*}

\begin{ExampleDdHTML*}{}{}
  \begin{lstlisting}[style=HTMLDraculaDark]  
  <!DOCTYPE html>
  <html lang="en">
	<head>
		<title>Hello!</title>
		<style>
			#container {
			    <!--indique qu'on veut affiche les elements a l'interieur 
				de container en mode flex-->	
				display: flex;


				<!-- indique qu'on veut que les elements "wrap"; s'il n'y a 
				pas suffisammment d'espace
				sur une ligne, placer les prochains elements dans le container 
				sur la ligne suivante. 
				flex-wrap: wrap;
				}
			
			<!-- selectionne tous les elements div qui sont enfant du container-->
			#container > div {
				background-color: springgreen;
				font-size: 20px;
				margin: 20px;
				padding: 20px;
				width: 200px;
			}
	</head>
	<body>
		<div id="container">
			<div class="grid-item"></div>
			<div>1.This is some sample text inside of div to demo Flexbox.</div>
			<div>2.This is some sample text inside of div to demo Flexbox.</div>
			<div>3.This is some sample text inside of div to demo Flexbox.</div>
			<div>4.This is some sample text inside of div to demo Flexbox.</div>
		</div>
	</body>
	</html>
  \end{lstlisting}
\end{ExampleDdHTML*}



\begin{Remarque}{Créer un \textit{flexbox}}{}
Les éléments qu'on désire contrôler par \textit{flexbox} doivent être placé à l'extérieur d'un 
\texttt{\footnotesize{<div> </div>}}. On donne à ce div un  \texttt{\footnotesize{id}} qu'on pourra sélectionner
lorsqu'on modifiera son style. 
\end{Remarque}




\section{Grid}
\begin{Syntaxe*}{Grid}{}
L'utilisation de grid est pertinente lorsqu'on veut organiser les éléments en grilles. 
\end{Syntaxe*}


\begin{ExampleDdHTML*}{}{}
  \begin{lstlisting}[style=HTMLDraculaDark] 
  <!DOCTYPE html>
  <html lang="en">
	<head>
		<title>Hello!</title>
		<style>
			#grid {
			    <!--indique qu'on veut affiche les elements a l'interieur 
				de container en mode grid-->	
				display: grid;


				<!-- indique l'espace qu'on veut entre chaque element de la grille 
				grid-column-gap: 20px;
				grid-row-gap: 10px;

				grid-template-columns: 200px 200px auto;
				}
			
			<!-- selectionne tous les elements div qui faisant parti de la classe "grid-item"-->
			.grid-item {
				background-color: white;
				font-size: 20px
				padding: 20px
				text-align: center;


			}
	</head>
	<body>
		<div id="container">
			<div class="grid-item">1. Un element</div>
			<div class="grid-item">2. Un element</div>
			<div class="grid-item">3. Un element</div>

			<div class="grid-item">1. Un element</div>
			<div class="grid-item">2. Un element</div>
			<div class="grid-item">3. Un element</div>	
			
		</div>
	</body>
	</html>
  \end{lstlisting}
\end{ExampleDdHTML*}

\begin{Remarque}{Créer un \textit{grid}}{}
Les éléments qu'on désire contrôle par \textit{grid} sont placé dans un \texttt{\footnotesize{<div> </div>}}. 
On donne à ce div un id qu'on pourra sélectionner. La commande \texttt{\footnotesize{grid-template-column}} 
permet de préciser la quantité de colonnes et la largeur de chaqcune de celles-ci. Lorsqu'on spécifie "auto",
la colonne prend tous \textit{l'espace restant disponible}. 

\end{Remarque}





\section{Bootstrap}
\begin{note}
Bootstrap est une librairie qui contient des styles pour modifier l'apparence de certains éléments, 
simplement en référençant le CSS et sans avoir à définir le style soit-même. 

\end{note}

\begin{figure}[H]
	\centering
	\includegraphics[width=0.9\textwidth]{bootstrap}
	
\end{figure}

\begin{Syntaxe*}{Référencer bootstrap}{}
Pour utiliser bootstrap il faut, dans le document html concerné, fournir un lien référant à la librairie 
boostrap : \texttt{\footnotesize{<link rel="stylesheet" href="https://stackpath.bootstrapcdn.com/bootstrap/4.4>}}

\end{Syntaxe*}
\noindent Sur le site de bootstrap, on peut trouver une panoplie de composantes. Pour obtenir le style d'une composante de 
notre choix, il suffit d'attribuer à l'élément désiré la classe correspondante. 

\begin{DefG*}{Boostrap Column Model}{}
	Système d'organisation des éléments utilisé par boostrat pour s'assurer que la page web soit réactive.
	Boostrap divise une page en sections où chaque rangée est divisé en colonnes de 12 unités.
\end{DefG*}
\begin{Remarque*}{Fonctionnement de Bootstrap}{}
One popular bootstrap feature is their grid system. Bootstrap automatically splits a page into 12 columns, and we can decide how many columns an element takes up by adding the class col-x where x is a number between 1 and 12. For example, in the following page, we have a row of columns of equal width, and then a row where the center column is larger:
\end{Remarque*}
\begin{ExampleDdHTML*}{}{}
  \begin{lstlisting}[style=HTMLDraculaDark] 
  <html lang="en">
    <head>
        <title>My Web Page!</title>
        <link rel="stylesheet" href="https://stackpath.bootstrapcdn.com/bootstrap/4.4.1/css/bootstrap.min.css" integrity="sha384-Vkoo8x4CGsO3+Hhxv8T/Q5PaXtkKtu6ug5TOeNV6gBiFeWPGFN9MuhOf23Q9Ifjh" crossorigin="anonymous">
        <style>
            .row > div {
                padding: 20px;
                background-color: teal;
                border: 2px solid black;
            }
        </style>
    </head>
    <body>
        <div class="container">
            <div class="row">
                <div class="col-4">
                    This is a section.
                </div>
                <div class="col-4">
                    This is another section.
                </div>
                <div class="col-4">
                    This is a third section.
                </div>
            </div>
        </div>
        <br/>
        <div class="container">
            <div class="row">
                <div class="col-3">
                    This is a section.
                </div>
                <div class="col-6">
                    This is another section.
                </div>
                <div class="col-3">
                    This is a third section.
                </div>
            </div>
        </div>
    </body>
</html>  \end{lstlisting}
\end{ExampleDdHTML*}

\begin{note}
Le système boostrap gère l'apparence des éléments lorsque la fenêtre est modifiée. 	
En modifiant la taille de la fenêtre, les colonnes se comportent de façon réactive; la taille de chaque colonne
sera ajustée pour qu'elles apparaissent toutes sur la même ligne.
\end{note}
\begin{Remarque*}{Taille de colonnne en fonction de la taille de l'écran}{}
To improve mobile-responsiveness, bootstrap also allows us to specify column sizes that differ depending on the screen size. In the following example, we use col-lg-3 to show that an element should take up 3 columns on a large screen, and col-sm-6 to show an element should take up 6 columns when the screen is small:
\end{Remarque*}

\begin{ExampleDdHTML*}{}{}
\begin{lstlisting}[style=HTMLDraculaDark]
	<!DOCTYPE html>
<html lang="en">
    <head>
        <title>My Web Page!</title>
        <link rel="stylesheet" href="https://stackpath.bootstrapcdn.com/bootstrap/4.4.1/css/bootstrap.min.css" integrity="sha384-Vkoo8x4CGsO3+Hhxv8T/Q5PaXtkKtu6ug5TOeNV6gBiFeWPGFN9MuhOf23Q9Ifjh" crossorigin="anonymous">
        <style>
            .row > div {
                padding: 20px;
                background-color: teal;
                border: 2px solid black;
            }
        </style>
    </head>
    <body>
        <div class="container">
            <div class="row">
                <div class="col-lg-3 col-sm-6">
                    This is a section.
                </div>
                <div class="col-lg-3 col-sm-6">
                    This is another section.
                </div>
                <div class="col-lg-3 col-sm-6">
                    This is a third section.
                </div>
                <div class="col-lg-3 col-sm-6">
                    This is a fourth section.
                </div>
            </div>
        </div>
    </body>
</html>
\end{lstlisting}
\end{ExampleDdHTML*}

%====================================================================
% 
%====================================================================


\chapter{SASS}
\section{Création de SASS}
\begin{DefG}{SCSS}{}
	Il s'agit d'un langage qui est une extansion de CSS. SCCS possède toutes les fonctionnalités de CSS et plus encoere—notamment, la possibilité de créer des variables.
\end{DefG}


\begin{Syntaxe*}{Création d'un fichier SASS}{}
Les fichiers SASS ont comme extansion \texttt{\footnotesize{.scss}} au lieu de \texttt{\footnotesize{CSS}}.  
Chaque variable débute avec le symbole \texttt{\footnotesize{\$}}
\end{Syntaxe*}

 


\begin{ExampleDdHTML*}{}{}
  \begin{lstlisting}[style=HTMLDraculaDark] 
  <!--Fichier HTML-->
  <!DOCTYPE html>
  <html lang="en">
	<head>
		<title>Formulaire</title>
		<link rel="stylesheet" href="variable.scss">	
		<style>
			.row > div {
				padding: 20px;
				background-color: teal;
				border:2px solid black;
			}
		</style>
	</head>
	<body>
		<div class="container">
			<div class="row">
				<div class="col-3">This is a section</div>
				<div class="col-3">This is another section</div>
				<div class="col-3">This is a third section</div>
			</div>
		</div>
	</body>
  </html>
\end{lstlisting}

\begin{lstlisting}[style=CSSDraculaDark] 
  /* Fichier SASS */
  $variable:red; 
  /* on assigne la couleur rouge a la virable "variable" */ 	
  ul {
		font-size:14px;
		color:variable;	/* On utilise la variable "variable" pour donner une couleur rouge au texte contenu dans <ul> */ 
  }
  ul {
		font-size:18px;
		color:variable;
  }
  \end{lstlisting}
\end{ExampleDdHTML*}


\section{Compilation de SCSS}
\noindent 
Le navigateur ne comprends pas le langage SCSS. Lorsqu'on écrit le code CSS, il faut le compiler en CSS. Dans le
terminal on peut utiliser la commande \texttt{\footnotesize{sass nomFichier.scss:nomFichierFin.css}} où 
\texttt{\footnotesize{nomFichier}} est le bom du fichier scsss et 
\texttt{\footnotesize{nomFichierVer}} est le nom du fichier final vers lequel on veut faire la conversion 

\begin{Remarque*}{Recompilation du fichier SCSS}{}
Après chaque modification du fichier SCSS, il faut le compiler à nouveau pour que les changements soient appliqués.
Lorsqu'on travail à partir du terminal, on peut utiliser la commande \texttt{\footnotesize{--watch nomFichier.scss:nomFichierVer.css}} pour que la compilation se fasse automatiquement.  
\end{Remarque*}



\section{Simplification de la sélection d'éléments}
\begin{Syntaxe*}{Sélecteur simplifié}{}
	Il est possible der simplifier la sélection d'élément structurant les éléments dans le fichier SCSS
	pour mettre en évidance la relation entre les éléments. Dans l'exemple suivant, seuls les éléments \entouree[gray]{\texttt{\footnotesize{p}}} et \entouree[gray]{\texttt{\footnotesize{ul}}} qui sont dans un 
	\entouree[gray]{\texttt{\footnotesize{div}}} sont modifié.
\end{Syntaxe*}


\begin{ExampleDdHTML*}{}{}
  \begin{lstlisting}[style=CSSDraculaDark]
  /* Fichier SASS : organisation heriarchique facilitant la selection */
div {
    font-size: 18px;

    p {
        color: blue;
    }

    ul {
        color: green;
    }
}
  \end{lstlisting}
    \begin{lstlisting}[style=CSSDraculaDark]
  
  /* Fichier CSS correspondant : addition automatique des selecteur en s'insipirant de la hirerachie dictee par SASS */
div {
    font-size: 18px;
}

div p {
    color: blue;
}

div ul {
    color: green;
}
	\end{lstlisting}

\end{ExampleDdHTML*}



\section{Héritage}
\noindent Il est possible de donner des propriétés à certains éléments et de faire en sorte que d'autres éléments héritent
de ces propriétés. L'exemple suivant montre que tous les éléments \entouree[gray]{\texttt{\footnotesize{success}}}
héritent des propriétés de la classe \entouree[gray]{\texttt{\footnotesize{message}}}. 


\begin{ExampleDdHTML*}{}{}
  \begin{lstlisting}[style=CSSDraculaDark] 
    %message {
		font-family:sans-serif;
		font-size:18px;
		font-weight:bold;
		border:1px solid black;
		padding:20px;
		margin:20px;
	}

	success {
		@extend %message;	  /* success herite des proprietes de message */
		background-color: green;
	}\end{lstlisting}
\end{ExampleDdHTML*}



\begin{ExampleDdHTML*}{Résumé des principales balises}{}
\begin{lstlisting}[style=HTMLDraculaDark]
<!DOCTYPE html>
<html lang="en">
    <head>
        <title>HTML Elements</title>
    </head>
    <body>
        <!-- We can create headings using h1 through h6 as tags. -->
        <h1>A Large Heading</h1>
        <h2>A Smaller Heading</h2>
        <h6>The Smallest Heading</h6>

        <!-- The strong and i tags give us bold and italics respectively. -->
        A <strong>bold</strong> word and an <i>italicized</i> word!

        <!-- We can link to another page (such as cs50's page) using a. -->
        View the <a href="https://cs50.harvard.edu/">CS50 Website</a>!

        <!-- We used ul for an unordered list and ol for an ordered one. both ordered and unordered lists contain li, or list items. -->
        An unordered list:
        <ul>
            <li>foo</li>
            <li>bar</li>
            <li>baz</li>
        </ul>
        An ordered list:
        <ol>
            <li>foo</li>
            <li>bar</li>
            <li>baz</li>
        </ol>

        <!-- Images require a src attribute, which can be either the path to a file on your computer or the link to an image online. It also includes an alt attribute, which gives a description in case the image can't be loaded. -->
        An image:
        <img src="../../images/duck.jpeg" alt="Rubber Duck Picture">
        <!-- We can also see above that for some elements that don't contain other ones, closing tags are not necessary. -->

        <!-- Here, we use a br tag to add white space to the page. -->
        <br/> <br/>

        <!-- A few different tags are necessary to create a table. -->
        <table>
            <thead>
                <th>Ocean</th>
                <th>Average Depth</th>
                <th>Maximum Depth</th>
            </thead>
            <tbody>
                <tr>
                    <td>Pacific</td>
                    <td>4280 m</td>
                    <td>10911 m</td>
                </tr>
                <tr>
                    <td>Atlantic</td>
                    <td>3646 m</td>
                    <td>8486 m</td>
                </tr>
            </tbody>
        </table>
    </body>
<html>	
\end{lstlisting}
\end{ExampleDdHTML*}

\end{document}
