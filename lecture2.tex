\documentclass{report}
\usepackage[french]{babel}

% Permet d'ajuster la taille des marges et de la distance pour les footer
\usepackage[tmargin=2cm,rmargin=1in,lmargin=1in,margin=0.85in,bmargin=2cm,footskip=.2in]{geometry}

% Permet d'optimiser l'affichage de différents symboles et formules mathématiques
\usepackage{amsmath,amsfonts,amsthm,amssymb,mathtools}

\usepackage{svg}

% Modifie l'apparence des nombre en mathmode et textmode
\usepackage[varbb]{newpxmath}

% Modifier l'apparence des fractions
\usepackage{xfrac}

% Permet de rayer (barrer) l'argument avec la touche
% \cancel{} \bcancel{} ou \xcancel{}
\usepackage[makeroom]{cancel}

% Extension du package amsmath; corrige certains bugs et déficiences de son prédecesseur
\usepackage{mathtools}

% This package provides most of the flexibility you may want to customize the three basic list
% environments (enumerate, itemize and description)
\usepackage{bookmark} 

% Réorganiser les théorèmes et Lemmes. Usage complexe. 
% Référence : https://ctan.math.illinois.edu/macros/latex/contrib/theoremref/theoremref-doc.pdf
\hypersetup{hidelinks}
\usepackage{hyperref,theoremref} 

% Fournit un environnement pour créer des boîtes colorées
\usepackage[most,many,breakable]{tcolorbox}


%\newcommand\mycommfont[1]{\footnotesize\ttfamily\textcolor{blue}{#1}}\SetCommentSty{mycommfont}

%\newcommand{\incfig}[1]{%\def\svgwidth{\columnwidth}\import{./figures/}{#1.pdf_tex}}
\newcommand{\arc}[1]{\wideparen{#1}}

%Pour colorer les lignes séparatrices de tableaux
\usepackage{colortbl}
\usepackage{tikzsymbols}

\usepackage{framed}
\usepackage{titletoc}
\usepackage{etoolbox}
\usepackage{lmodern}
\usepackage{tabularx}
\usepackage{enumitem}
\usepackage{amsthm}
%==========================================================================================
\usepackage{libris} 
\usepackage{etoolbox}
\usepackage[export]{adjustbox}% for positioning figures

\makeatletter
% Force le chapitre suivant sur la ligne succedant la fin du 
% chapitre précédent
\patchcmd{\chapter}{\if@openright\cleardoublepage\else\clearpage\fi}{}{}{}
\makeatother
\usepackage[Glenn]{fncychap}


%boîte de couleur grise
\tcbset{
  graybox/.style={
    colback=gray!20,
    colframe=black,
    sharp corners=downhill,
    boxrule=1pt,
    left=5pt,
    right=5pt,
    top=5pt,
    bottom=5pt,
    boxsep=0pt,
	 % <-- add four values for each corner
  }
}
\newtcolorbox{graybox}{graybox}

%==========================================================================================



\usepackage{xcolor}
\usepackage{varwidth}
\usepackage{varwidth}
\usepackage{etoolbox}
%\usepackage{authblk}
\usepackage{nameref}
\usepackage{multicol,array}
\usepackage{tikz-cd}
\usepackage[ruled,vlined,linesnumbered]{algorithm2e}
\usepackage{comment} % enables the use of multi-line comments (\ifx \fi) 
\usepackage{import}
\usepackage{xifthen}
\usepackage{pdfpages}
\usepackage{transparent}


%\usepackage[french]{babel}
\usepackage{listings} % pour écrire du code dans un environnement
\lstset{
  basicstyle=\ttfamily,
  columns=fullflexible,
  keepspaces=true
}
\usepackage{caption}
\usepackage{float} % Pour forcer les images au bon endroit



\usepackage[T1]{fontenc}
\usepackage{csquotes}
%%%%%%%%%%%%%%%%%%%%%%%%%%%%%%%%%%%%%%%%%%%%%%%%%%%%%%%%%%%%%%%%%%%%%%%%%%%%%%%%%%%%%%%%%%%%%%%%%
%									ENSEMBLE DE COULEURS
%%%%%%%%%%%%%%%%%%%%%%%%%%%%%%%%%%%%%%%%%%%%%%%%%%%%%%%%%%%%%%%%%%%%%%%%%%%%%%%%%%%%%%%%%%%%%%%%%

\definecolor{myg}{RGB}{56, 140, 70}
\definecolor{myb}{RGB}{45, 111, 177}

\definecolor{mygbg}{RGB}{235, 253, 241}


\definecolor{myr}{RGB}{199, 68, 64}
\definecolor{mytheorembg}{HTML}{F2F2F9}
\definecolor{mytheoremfr}{HTML}{00007B}
\definecolor{mylenmabg}{HTML}{FFFAF8}
\definecolor{mylenmafr}{HTML}{983b0f}
\definecolor{mypropbg}{HTML}{f2fbfc}
\definecolor{mypropfr}{HTML}{191971}
\definecolor{myexamplebg}{HTML}{F2FBF8}
\definecolor{myexamplefr}{HTML}{88D6D1}
\definecolor{myexampleti}{HTML}{2A7F7F}
\definecolor{mydefinitbg}{HTML}{E5E5FF}
\definecolor{mydefinitfr}{HTML}{3F3FA3}
\definecolor{notesgreen}{RGB}{0,162,0}
\definecolor{myp}{RGB}{197, 92, 212}
\definecolor{mygr}{HTML}{2C3338}
\definecolor{myred}{RGB}{127,0,0}
\definecolor{myyellow}{RGB}{169,121,69}
\definecolor{myexercisebg}{HTML}{F2FBF8}
\definecolor{myexercisefg}{HTML}{88D6D1}
\definecolor{myred}{RGB}{127,0,0}
\definecolor{myyellow}{RGB}{169,121,69}

\definecolor{blue}{HTML}{008ED7}
\definecolor{mygray}{gray}{0.75}
\definecolor{lightBlue}{RGB}{235, 245, 255}
\definecolor{tcbcolred}{RGB}{255,0,0}
\definecolor{myGreen}{HTML}{009900}

% command to circle a text
\newtcbox{\entoure}[1][red]{on line,
	arc=3pt,colback=#1!10!white,colframe=#1!50!black,
	before upper={\rule[-3pt]{0pt}{10pt}},boxrule=1pt,
	boxsep=0pt,left=2pt,right=2pt,top=1pt,bottom=.5pt}
% command for the circle for the number of part entries
\newcommand\Circle[1]{\tikz[overlay,remember picture]
	\node[draw,circle, text width=18pt,line width=1pt] {#1};}

\newtcbox{\entouree}[1][red]{on line,
	arc=3pt,colback=#1!10!white,colframe=#1!50!white,
	before upper={\rule[-3pt]{0pt}{10pt}},boxrule=1pt,
	boxsep=0pt,left=2pt,right=2pt,top=1pt,bottom=.5pt}

\newcommand{\shellcmd}[1]{\\\indent\indent\texttt{\footnotesize\# #1}\\}

%=====================================================================
\patchcmd{\tableofcontents}{\contentsname}{\sffamily\contentsname}{}{}
% patching of \@part to typeset the part number inside a framed box in its own line
% and to add color
\makeatletter
\patchcmd{\@part}
  {\addcontentsline{toc}{part}{\thepart\hspace{1em}#1}}
  {\addtocontents{toc}{\protect\addvspace{20pt}}
    \addcontentsline{toc}{part}{\huge{\protect\color{myyellow}%
      \setlength\fboxrule{2pt}\protect\Circle{%
        \hfil\thepart\hfil%
      }%
    }\\[2ex]\color{myred}\sffamily#1}}{}{}

%\patchcmd{\@part}
%  {\addcontentsline{toc}{part}{\thepart\hspace{1em}#1}}
%  {\addtocontents{toc}{\protect\addvspace{20pt}}
%    \addcontentsline{toc}{part}{\huge{\protect\color{myyellow}%
%      \setlength\fboxrule{2pt}\protect\fbox{\protect\parbox[c][1em][c]{1.5em}{%
%        \hfil\thepart\hfil%
%      }}%
%    }\\[2ex]\color{myred}\sffamily#1}}{}{}
\makeatother
% this is the environment used to typeset the chapter entries in the ToC
% it is a modification of the leftbar environment of the framed package
\renewenvironment{leftbar}
  {\def\FrameCommand{\hspace{6em}%
    {\color{myyellow}\vrule width 2pt depth 6pt}\hspace{1em}}%
    \MakeFramed{\parshape 1 0cm \dimexpr\textwidth-6em\relax\FrameRestore}\vskip2pt%
  }
 {\endMakeFramed}

% using titletoc we redefine the ToC entries for parts, chapters, sections, and subsections
\titlecontents{part}
  [0em]{\centering}
  {\contentslabel}
  {}{}
\titlecontents{chapter}
  [0em]{\vspace*{2\baselineskip}}
  {\parbox{4.5em}{%
    \hfill\Huge\sffamily\bfseries\color{myred}\thecontentspage}%
   \vspace*{-2.3\baselineskip}\leftbar\textsc{\small\chaptername~\thecontentslabel}\\\sffamily}
  {}{\endleftbar}
\titlecontents{section}
  [8.4em]
  {\sffamily\contentslabel{3em}}{}{}
  {\hspace{0.5em}\nobreak\itshape\color{myred}\normalfont\contentspage}
\titlecontents{subsection}
  [8.4em]
  {\sffamily\contentslabel{3em}}{}{}  
  {\hspace{0.5em}\nobreak\itshape\color{myred}\contentspage}
%==========================================================================

%PYTHON LSTLISTING STYLE

% Define colors
\definecolor{Pgruvbox-bg}{HTML}{282828}
\definecolor{Pgruvbox-fg}{HTML}{ebdbb2}
\definecolor{Pgruvbox-red}{HTML}{fb4934}
\definecolor{Pgruvbox-green}{HTML}{b8bb26}
\definecolor{Pgruvbox-yellow}{HTML}{fabd2f}
\definecolor{Pgruvbox-blue}{HTML}{83a598}
\definecolor{Pgruvbox-purple}{HTML}{d3869b}
\definecolor{Pgruvbox-aqua}{HTML}{8ec07c}

% Define Python style
\lstdefinestyle{PythonGruvbox}{
	language=Python,
	identifierstyle=\color{lst-fg},
	basicstyle=\ttfamily\color{Pgruvbox-fg},
	keywordstyle=\color{Pgruvbox-yellow},
	keywordstyle=[2]\color{Pgruvbox-blue},
	stringstyle=\color{Pgruvbox-green},
	commentstyle=\color{Pgruvbox-aqua},
	backgroundcolor=\color{Pgruvbox-bg},
	%frame=tb,
	rulecolor=\color{Pgruvbox-fg},
	showstringspaces=false,
	keepspaces=true,
	captionpos=b,
	breaklines=true,
	tabsize=4,
	showspaces=false,
	numbers=left,
	numbersep=5pt,
	numberstyle=\tiny\color{gray},
	showtabs=false,
	columns=fullflexible,
	morekeywords={True,False,None},
	morekeywords=[2]{and,as,assert,break,class,continue,def,del,elif,else,except,exec,finally,for,from,global,if,import,in,is,lambda,nonlocal,not,or,pass,print,raise,return,try,while,with,yield},
	morecomment=[s]{"""}{"""},
	morecomment=[s]{'''}{'''},
	morecomment=[l]{\#},
	morestring=[b]",
	morestring=[b]',
	literate=
	{0}{{\textcolor{Pgruvbox-purple}{0}}}{1}
	{1}{{\textcolor{Pgruvbox-purple}{1}}}{1}
	{2}{{\textcolor{Pgruvbox-purple}{2}}}{1}
	{3}{{\textcolor{Pgruvbox-purple}{3}}}{1}
	{4}{{\textcolor{Pgruvbox-purple}{4}}}{1}
	{5}{{\textcolor{Pgruvbox-purple}{5}}}{1}
	{6}{{\textcolor{Pgruvbox-purple}{6}}}{1}
	{7}{{\textcolor{Pgruvbox-purple}{7}}}{1}
	{8}{{\textcolor{Pgruvbox-purple}{8}}}{1}
	{9}{{\textcolor{Pgruvbox-purple}{9}}}{1}
}
%====================================================================
% 
%====================================================================

% JAVA LSTLISTING STYLE IN Gruvbox Colorscheme
\definecolor{gruvbox-bg}{rgb}{0.282, 0.247, 0.204}
\definecolor{gruvbox-fg1}{rgb}{0.949, 0.898, 0.776}
\definecolor{gruvbox-fg2}{rgb}{0.871, 0.804, 0.671}
\definecolor{gruvbox-red}{rgb}{0.788, 0.255, 0.259}
\definecolor{gruvbox-green}{rgb}{0.518, 0.604, 0.239}
\definecolor{gruvbox-yellow}{rgb}{0.914, 0.808, 0.427}
\definecolor{gruvbox-blue}{rgb}{0.353, 0.510, 0.784}
\definecolor{gruvbox-purple}{rgb}{0.576, 0.412, 0.659}
\definecolor{gruvbox-aqua}{rgb}{0.459, 0.631, 0.737}
\definecolor{gruvbox-gray}{rgb}{0.518, 0.494, 0.471}

\definecolor{lst-bg}{RGB}{45, 45, 45}
\definecolor{lst-fg}{RGB}{220, 220, 204}
\definecolor{lst-keyword}{RGB}{215, 186, 125}
\definecolor{lst-comment}{RGB}{117, 113, 94}
\definecolor{lst-string}{RGB}{163, 190, 140}
\definecolor{lst-number}{RGB}{181, 206, 168}
\definecolor{lst-type}{RGB}{218, 142, 130}


\lstdefinestyle{JavaGruvbox}{
	language=Java,
	basicstyle=\ttfamily\color{lst-fg},
	keywordstyle=\color{lst-keyword},
	keywordstyle=[2]\color{lst-type},
	commentstyle=\itshape\color{lst-comment},
	stringstyle=\color{lst-string},
	numberstyle=\color{lst-number},
	backgroundcolor=\color{lst-bg},
	%frame=tb,
	rulecolor=\color{gruvbox-aqua},
	showstringspaces=false,
	keepspaces=true,
	captionpos=b,
	breaklines=true,
	tabsize=4,
	showspaces=false,
	showtabs=false,
	columns=fullflexible,
	morekeywords={var},
	morekeywords=[2]{boolean, byte, char, double, float, int, long, short, void},
	morecomment=[s]{/}{/},
	morecomment=[l]{//},
	morestring=[b]",
	morestring=[b]',
	numbers=left,
	numbersep=5pt,
	numberstyle=\tiny\color{gray},
}



%====================================================================
% 
%====================================================================


% Define Dracula color scheme for Java
\definecolor{draculawhite-background}{RGB}{237, 239, 252}
\definecolor{draculawhite-comment}{RGB}{98, 114, 164}
\definecolor{draculawhite-keyword}{RGB}{189, 147, 249}
\definecolor{draculawhite-string}{RGB}{152, 195, 121}
\definecolor{draculawhite-number}{RGB}{249, 189, 89}
\definecolor{draculawhite-operator}{RGB}{248, 248, 242}

% Define JavaDraculaWhite lstlisting environment
\lstdefinestyle{JavaDraculaWhite}{
    language=Java,
    backgroundcolor=\color{draculawhite-background},
    commentstyle=\itshape\color{draculawhite-comment},
    keywordstyle=\color{draculawhite-keyword},
    stringstyle=\color{draculawhite-string},
    basicstyle=\ttfamily\small\color{black},
    identifierstyle=\color{black},
    keywordstyle=\color{draculawhite-keyword}\bfseries,
    morecomment=[s][\color{draculawhite-comment}]{/**}{*/},
    showstringspaces=false,
    showspaces=false,
    breaklines=true,
    frame=single,
    rulecolor=\color{draculawhite-operator},
    tabsize=4,  
	numbers=left,
	numbersep=4pt,
	numberstyle=\ttfamily\tiny\color{gray}
}
%====================================================================
% 
%====================================================================
% Define PythonDraculaWhite lstlisting environment 
\definecolor{draculawhite-bg}{HTML}{FAFAFA}
\definecolor{draculawhite-fg}{HTML}{282A36}
\definecolor{pdraculawhite-keyword}{HTML}{BD93F9}

\definecolor{pdraculawhite-comment}{HTML}{6272A4}
\definecolor{draculawhite-number}{HTML}{FF79C6}


\lstdefinestyle{PythonDraculaWhite}{
    language=Python,
    basicstyle=\ttfamily\small\color{draculawhite-fg},
    backgroundcolor=\color{draculawhite-background},
    keywordstyle=\color{orange}\bfseries,
    stringstyle=\color{draculawhite-string},
    commentstyle=\color{pdraculawhite-comment}\itshape,
    numberstyle=\color{draculawhite-number},
    showstringspaces=false,
	showspaces=false,
    breaklines=true,
	frame=single,
	rulecolor=\color{draculawhite-operator}, 
    tabsize=4,
    morekeywords={as,with,1,2,3,4, 5,6,7,8,9,True,False,@media},
    %escapeinside={(*@}{@*)},
    numbers=left,
    numbersep=5pt,
    %xleftmargin=15pt,
    %framexleftmargin=15pt,
    %framexrightmargin=0pt,
    %framexbottommargin=0pt,
    %framextopmargin=0pt,
    %rulecolor=\color{draculawhite-fg},
    %frame=tb,
    %aboveskip=0pt,
    %belowskip=0pt,
    %captionpos=b,
	numberstyle=\ttfamily\tiny\color{gray} 
}
%====================================================================
% 
%====================================================================

% Define colors for HTML langage
\definecolor{html-orange}{HTML}{FF5733}
\definecolor{html-yellow}{HTML}{F0E130}
\definecolor{html-green}{HTML}{50FA7B}
\definecolor{html-blue}{HTML}{5AFBFF}
\definecolor{html-purple}{HTML}{BD93F9}
\definecolor{html-pink}{HTML}{FF80BF}
\definecolor{html-gray}{HTML}{6272A4}
\definecolor{html-white}{HTML}{F8F8F2}

% Defines a new HTML5 langage that extend on the html langange
\lstdefinestyle{HTMLDraculaWhite}{
  language=HTML,
  backgroundcolor=\color{html-white},
  basicstyle=\ttfamily\color{html-gray},
  keywordstyle=\color{html-blue},
  stringstyle=\color{html-orange},
  commentstyle=\color{html-green},
  tagstyle=\color{html-yellow},
  moredelim=[s][\color{html-pink}]{<!--}{-->},
  moredelim=[s][\color{html-purple}]{\{}{\}},
  showstringspaces=false,
  tabsize=2,
  breaklines=true,
  columns=fullflexible,
  %frame=single,
  framexleftmargin=5mm,
  xleftmargin=10mm,
  numbers=left,
  numberstyle=\tiny\color{html-gray},
  escapeinside={<@}{@>},
}

%====================================================================
% 
%====================================================================
% Define the colors needed for the HTMLDraculaDark environment
\definecolor{htmltag}{HTML}{ff79c6}
\definecolor{htmlattr}{HTML}{f1fa8c}
\definecolor{htmlvalue}{HTML}{bd93f9}
\definecolor{htmlcomment}{HTML}{6272a4}
%\definecolor{htmltext}{HTML}{f8f8f2}
\definecolor{htmltext}{HTML}{401E31}
\definecolor{htmlbackground}{HTML}{282a36}
\definecolor{comphtmlbackground}{HTML}{8093FF}
%\definecolor{htmlbackground}{HTML}{4D5169}

% Define the HTMLDraculaDark environment
\lstdefinestyle{HTMLDraculaDark}{
    basicstyle=\bfseries\ttfamily\color{htmltext},
    commentstyle=\itshape\color{htmlcomment},
    keywordstyle=\bfseries\color{htmltag},
    stringstyle=\color{htmlvalue},
    emph={DOCTYPE,html,head,body,div,span,a,script},
    emphstyle={\color{htmltag}\bfseries},
    sensitive=true,
    showstringspaces=false,
    backgroundcolor=\color{white},
    %frame=tb,
    language=HTML,
    tabsize=4,
    breaklines=true,
    breakatwhitespace=true,
    numbers=left,
    numbersep=10pt,
    numberstyle=\small\bfseries\ttfamily\color{htmlcomment},
    escapeinside={<@}{@>},
	rulecolor=\color{htmlbackground},
	xleftmargin=20pt,
	% Add a vertical line for opening and closing tags
    %frame=lines,
    framesep=2pt,
    framexleftmargin=4pt,
    % Add a vertical line for closing tags that go through multiple lines
    breaklines=true,
    postbreak=\mbox{\textcolor{gray}{$\hookrightarrow$}\space},
    showlines=true,
	% Add a rule to apply commentstyle to keywords inside comments
    moredelim=[s][\itshape\color{htmlcomment}]{<!--}{-->},
    morekeywords={id,class,type,name,value,placeholder,checked,src,href,alt}
}
%====================================================================
% 
%====================================================================
\lstdefinelanguage{CSS}{
  keywords={color,background,font-weight,style,family,size,text-decoration,font-family,text-align,font-size,font-weight,border, padding, margin,},
  %keywordstyle=\color{CSSSelector},
  %keywords={[2]font-size},
  %keywordstyle={[2]\color{CSSProperty}},
  moredelim=*[s][\color{CSSValue}]{:}{;},
  moredelim=*[s][\color{orange}]{\ }{,},
  alsoletter=-,
  sensitive=true,
  comment=[l]{//},
  morecomment=[s]{/*}{*/},
  commentstyle=\color{CSSComment},
  stringstyle=\color{CSSValue},
  morestring=[b]',
  morestring=[b]",
}



\definecolor{CSSComment}{HTML}{6272A4}
\definecolor{CSSSelector}{HTML}{50FA7B}
\definecolor{CSSProperty}{HTML}{FF79C6}
\definecolor{CSSValue}{HTML}{BD93F9}

\lstdefinestyle{CSSDraculaDark}{
  language=css,
  basicstyle=\bfseries\ttfamily,
  commentstyle=\itshape\color{CSSComment},
  keywordstyle=\bfseries\color{CSSProperty},
  %keywordstyle={[2]\color{blue}},
  %keywordstyle={[3]\color{blue}},
  %keywordstyle={[4]\color{blue}},
  stringstyle=\color{CSSValue},
  morecomment=[s]{/*}{*/},
  morekeywords={color,background},
  morekeywords={[2]font-size},
  frame=tb,
  framesep=4pt,
  framerule=0pt,
  showspaces=false,
  showstringspaces=false,
  breaklines=true,
  breakatwhitespace=true,
  tabsize=4,
  numbers=left,
  numberstyle=\small\bfseries\ttfamily\color{htmlcomment} ,
  numbersep=10pt,
  xleftmargin=20pt,
  aboveskip=5pt,
  belowskip=5pt,
}


%====================================================================
% 
%====================================================================
% Crée un environnement "Theorem" numéroté en fonction du document
\tcbuselibrary{theorems,skins,hooks} 
\newtcbtheorem{Theorem}{Théorème}
{%
	enhanced,
	breakable,
	colback = mytheorembg,
	frame hidden,
	boxrule = 0sp,
	borderline west = {2pt}{0pt}{mytheoremfr},
	sharp corners,
	detach title,
	before upper = \tcbtitle\par\smallskip,
	coltitle = mytheoremfr,
	fonttitle = \bfseries\fontfamily{lmss}\selectfont,
	description font = \mdseries\fontfamily{lmss}\selectfont,
	separator sign none,
	segmentation style={solid, mytheoremfr},
}
{thm}

% Crée un environnement "Preuve" numéroté en fonction du document
\tcbuselibrary{theorems,skins,hooks}
\newtcbtheorem{Preuve}{Preuve.}
{
	enhanced,
	breakable,
	colback=white,
	frame hidden,
	boxrule = 0sp,
	borderline west = {2pt}{0pt}{mytheoremfr},
	sharp corners,
	detach title,
	before upper = \tcbtitle\par\smallskip,
	coltitle = mytheoremfr,
	description font=\fontfamily{lmss}\selectfont,
	fonttitle=\fontfamily{lmss}\selectfont\bfseries,
	separator sign none,
	segmentation style={solid, mytheoremfr},
}
{th}


% Crée un environnement "Preuve" numéroté en fonction du document
\tcbuselibrary{theorems,skins,hooks}
\newtcbtheorem{Explication}{Explication}
{
	enhanced,
	breakable,
	colback=white,
	frame hidden,
	boxrule = 0sp,
	borderline west = {2pt}{0pt}{mytheoremfr},
	sharp corners,
	detach title,
	before upper = \tcbtitle\par\smallskip,
	coltitle = mytheoremfr,
	description font=\fontfamily{lmss}\selectfont,
	fonttitle=\fontfamily{lmss}\selectfont\bfseries,
	separator sign none,
	segmentation style={solid, mytheoremfr},
}
{th}




% Crée un environnement "Example" numéroté en fonction du document
\tcbuselibrary{theorems,skins,hooks}
\newtcbtheorem{Example}{Exemple.}
{
	enhanced,
	breakable,
	colback=lightBlue,
	frame hidden,
	boxrule = 0sp,
	borderline west = {2pt}{0pt}{myb},
	sharp corners,
	detach title,
	before upper = \tcbtitle\par\smallskip,
	coltitle = myb,
	description font=\fontfamily{lmss}\selectfont,
	fonttitle=\fontfamily{lmss}\selectfont\bfseries,
	separator sign none,
	segmentation style={solid, mytheoremfr},
}
{th}



% Crée un environnement "EExample" numéroté en fonction du document
\tcbuselibrary{theorems,skins,hooks}
\newtcbtheorem{EExample}{Exemple.}
{
	enhanced,
	breakable,
	colback=white,
	frame hidden,
	boxrule = 0sp,
	borderline west = {2pt}{0pt}{myb},
	sharp corners,
	detach title,
	before upper = \tcbtitle\par\smallskip,
	coltitle = myb,
	description font=\mdseries\fontfamily{lmss}\selectfont,
	fonttitle=\fontfamily{lmss}\selectfont\bfseries,
	separator sign none,
	segmentation style={solid, mytheoremfr},
}
{th}



% Crée un environnement "ExampleDdHTML" numéroté en fonction du document
\tcbuselibrary{theorems,skins,hooks}
\newtcbtheorem{ExampleDdHTML}{Exemple.}
{
	enhanced,
	breakable,
	colback=white,
	frame hidden,
	boxrule = 0sp,
	borderline west = {2pt}{0pt}{htmlbackground},
	sharp corners,
	detach title,
	before upper = \tcbtitle\par\smallskip,
	coltitle = htmlbackground,
	description font=\mdseries\fontfamily{lmss}\selectfont,
	fonttitle=\fontfamily{lmss}\selectfont\bfseries,
	separator sign none,
	segmentation style={solid, mytheoremfr},
}
{th}




% Crée un environnement "Lemme" numéroté en fonction du document
\tcbuselibrary{theorems,skins,hooks}
\newtcbtheorem{Lemme}{Lemme}
{
	enhanced,
	breakable,
	colback=mylenmabg,
	frame hidden,
	boxrule = 0sp,
	borderline west = {2pt}{0pt}{mylenmafr},
	sharp corners,
	detach title,
	before upper = \tcbtitle\par\smallskip,
	coltitle = mylenmafr,
	description font=\mdseries\fontfamily{lmss}\selectfont,
	fonttitle=\fontfamily{lmss}\selectfont\bfseries,
	separator sign none,
	segmentation style={solid, mytheoremfr},
}
{th}


\tcbuselibrary{theorems,skins,hooks}
\newtcbtheorem{PreuveL}{Preuve.}
{
	enhanced,
	breakable,
	colback=white,
	frame hidden,
	boxrule = 0sp,
	borderline west = {2pt}{0pt}{mylenmafr},
	sharp corners,
	detach title,
	before upper = \tcbtitle\par\smallskip,
	coltitle = mylenmafr,
	description font=\fontfamily{lmss}\selectfont,
	fonttitle=\fontfamily{lmss}\selectfont\bfseries,
	separator sign none,
	segmentation style={solid, mytheoremfr},
}
{th}


\newtcbtheorem{Remarque}{Remarque.}
{
	enhanced,
	breakable,
	colback=white,
	frame hidden,
	boxrule = 0sp,
	borderline west = {2pt}{0pt}{myb},
	sharp corners,
	detach title,
	before upper = \tcbtitle\par\smallskip,
	coltitle = myb,
	description font=\mdseries\fontfamily{lmss}\selectfont,
	fonttitle=\fontfamily{lmss}\selectfont\bfseries,
	separator sign none,
	segmentation style={solid, mytheoremfr},
}
{th}


\newtcbtheorem{DefG}{Définition}
{
	enhanced,
	breakable,
	colback=mygbg,
	frame hidden,
	boxrule = 0sp,
	borderline west = {2pt}{0pt}{myg},
	sharp corners,
	detach title,
	before upper = \tcbtitle\par\smallskip,
	coltitle = myg,
	description font=\mdseries\fontfamily{lmss}\selectfont,
	fonttitle=\fontfamily{lmss}\selectfont\bfseries,
	separator sign none,
	segmentation style={solid, mytheoremfr},
}
{th}



% Crée une boîte ayant la même couleur que l'environnement theorem.
\tcbuselibrary{theorems,skins,hooks}
\newtcolorbox{Theoremcon}
{%
	enhanced
	,breakable
	,colback = mytheorembg
	,frame hidden
	,boxrule = 0sp
	,borderline west = {2pt}{0pt}{mytheoremfr}
	,sharp corners
	,description font = \mdseries
	,separator sign none
}

% Crée un environnement "Definition" numéroté en fonction de la section
\newtcbtheorem[number within=chapter]{Definition}{Définition}{enhanced,
	before skip=2mm,after skip=2mm, colback=red!5,colframe=red!80!black,boxrule=0.5mm,
	attach boxed title to top left={xshift=1cm,yshift*=1mm-\tcboxedtitleheight}, varwidth boxed title*=-3cm,
	boxed title style={frame code={
			\path[fill=tcbcolback!10!red]
			([yshift=-1mm,xshift=-1mm]frame.north west)
			arc[start angle=0,end angle=180,radius=1mm]
			([yshift=-1mm,xshift=1mm]frame.north east)
			arc[start angle=180,end angle=0,radius=1mm];
			\path[left color=tcbcolback!10!myred,right color=tcbcolback!10!myred,
			middle color=tcbcolback!60!myred]
			([xshift=-2mm]frame.north west) -- ([xshift=2mm]frame.north east)
			[rounded corners=1mm]-- ([xshift=1mm,yshift=-1mm]frame.north east)
			-- (frame.south east) -- (frame.south west)
			-- ([xshift=-1mm,yshift=-1mm]frame.north west)
			[sharp corners]-- cycle;
		},interior engine=empty,
	},
	fonttitle=\bfseries,
	title={#2},#1}{def}

% Crée un environnement "definition" numéroté en fonction du Chapitre
\newtcbtheorem[number within=section]{definition}{Définition}{enhanced,
	before skip=2mm,after skip=2mm, colback=red!5,colframe=red!80!black,boxrule=0.5mm,
	attach boxed title to top left={xshift=1cm,yshift*=1mm-\tcboxedtitleheight}, varwidth boxed title*=-3cm,
	boxed title style={frame code={
			\path[fill=tcbcolback]
			([yshift=-1mm,xshift=-1mm]frame.north west)
			arc[start angle=0,end angle=180,radius=1mm]
			([yshift=-1mm,xshift=1mm]frame.north east)
			arc[start angle=180,end angle=0,radius=1mm];
			\path[left color=tcbcolback!60!black,right color=tcbcolback!60!black,
			middle color=tcbcolback!80!black]
			([xshift=-2mm]frame.north west) -- ([xshift=2mm]frame.north east)
			[rounded corners=1mm]-- ([xshift=1mm,yshift=-1mm]frame.north east)
			-- (frame.south east) -- (frame.south west)
			-- ([xshift=-1mm,yshift=-1mm]frame.north west)
			[sharp corners]-- cycle;
		},interior engine=empty,
	},
	fonttitle=\bfseries,
	title={#2},#1}{def}

\usetikzlibrary{arrows,calc,shadows.blur}
\tcbuselibrary{skins}
\newtcolorbox{note}[1][]{%
	enhanced jigsaw,
	colback=gray!20!white,%
	colframe=gray!80!black,
	size=small,
	boxrule=1pt,
	title=\textbf{Note : },
	halign title=flush center,
	coltitle=black,
	breakable,
	drop shadow=black!50!white,
	attach boxed title to top left={xshift=1cm,yshift=-\tcboxedtitleheight/2,yshifttext=-\tcboxedtitleheight/2},
	minipage boxed title=1.5cm,
	boxed title style={%
		colback=white,
		size=fbox,
		boxrule=1pt,
		boxsep=2pt,
		underlay={%
			\coordinate (dotA) at ($(interior.west) + (-0.5pt,0)$);
			\coordinate (dotB) at ($(interior.east) + (0.5pt,0)$);
			\begin{scope}
				\clip (interior.north west) rectangle ([xshift=3ex]interior.east);
				\filldraw [white, blur shadow={shadow opacity=60, shadow yshift=-.75ex}, rounded corners=2pt] (interior.north west) rectangle (interior.south east);
			\end{scope}
			\begin{scope}[gray!80!black]
				\fill (dotA) circle (2pt);
				\fill (dotB) circle (2pt);
			\end{scope}
		},
	},
	#1,
}


% Crée un environnement "qstion" 
\newtcbtheorem{qstion}{Question}{enhanced,
	breakable,
	colback=white,
	colframe=mygr,
	attach boxed title to top left={yshift*=-\tcboxedtitleheight},
	fonttitle=\bfseries,
	title={#2},
	boxed title size=title,
	boxed title style={%
		sharp corners,
		rounded corners=northwest,
		colback=tcbcolframe,
		boxrule=0pt,
	},
}{def}


% Pour créer un environnement "Liste" 

\tcbuselibrary{theorems,skins,hooks}
\newtcbtheorem[number within=section]{Liste}{Liste}
{%
	enhanced
	,breakable
	,colback = myp!10
	,frame hidden
	,boxrule = 0sp
	,borderline west = {2pt}{0pt}{myp!85!black}
	,sharp corners
	,detach title
	,before upper = \tcbtitle\par\smallskip
	,coltitle = myp!85!black
	,fonttitle = \bfseries\sffamily
	,description font = \mdseries
	,separator sign none
	,segmentation style={solid, myp!85!black}
}
{th}


\tcbuselibrary{theorems,skins,hooks}
\newtcbtheorem{Syntaxe}{Syntaxe.}
{%
	enhanced
	,breakable
	,colback = myp!10
	,frame hidden
	,boxrule = 0sp
	,borderline west = {2pt}{0pt}{myp!85!black}
	,sharp corners
	,detach title
	,before upper = \tcbtitle\par\smallskip
	,coltitle = myp!85!black
	,fonttitle = \bfseries\fontfamily{lmss}\selectfont 
	,description font = \mdseries\fontfamily{lmss}\selectfont 
	,separator sign none
	,segmentation style={solid, myp!85!black}
}
{th}



% Crée un environnement "Concept" numéroté en fonction du document
\tcbuselibrary{theorems,skins,hooks}
\newtcbtheorem{Concept}{Concept.}
{
	enhanced,
	breakable,
	colback=mylenmabg,
	frame hidden,
	boxrule = 0sp,
	borderline west = {2pt}{0pt}{mylenmafr},
	sharp corners,
	detach title,
	before upper = \tcbtitle\par\smallskip,
	coltitle = mylenmafr,
	description font=\mdseries\fontfamily{lmss}\selectfont,
	fonttitle=\fontfamily{lmss}\selectfont\bfseries,
	separator sign none,
	segmentation style={solid, mytheoremfr},
}
{th}


% Crée un environnement "codeEx" numéroté en fonction du document
\tcbuselibrary{theorems,skins,hooks}
\newtcbtheorem{codeEx}{Exemple.}
{
	enhanced,
	breakable,
	colback=white,
	frame hidden,
	boxrule = 0sp,
	borderline west = {2pt}{0pt}{gruvbox-bg},
	sharp corners,
	detach title,
	before upper = \tcbtitle\par\smallskip,
	coltitle = gruvbox-bg,
	description font=\mdseries\fontfamily{lmss}\selectfont,
	fonttitle=\fontfamily{lmss}\selectfont\bfseries,
	separator sign none,
	segmentation style={solid, mytheoremfr},
}
{th}


% Crée un environnement "codeEx" numéroté en fonction du document
\tcbuselibrary{theorems,skins,hooks}
\newtcbtheorem{codeRem}{Remarque.}
{
	enhanced,
	breakable,
	colback=white,
	frame hidden,
	boxrule = 0sp,
	borderline west = {2pt}{0pt}{gruvbox-bg},
	sharp corners,
	detach title,
	before upper = \tcbtitle\par\smallskip,
	coltitle = gruvbox-bg,
	description font=\mdseries\fontfamily{lmss}\selectfont,
	fonttitle=\fontfamily{lmss}\selectfont\bfseries,
	separator sign none,
	segmentation style={solid, mytheoremfr},
}
{th}


\tcbuselibrary{theorems,skins,hooks}










\title{\Huge{Harvard CS50W}\\Introduction à Python}
\author{\huge{Franz Girardin}}
\date{May $8^{th}$ 2023}
\usepackage{framed}

%====================================================================

%====================================================================
\begin{document}
\maketitle

\newpage
\pdfbookmark[section]{\contentsname}{toc}
\tableofcontents
\pagebreak
%====================================================================
% 
%====================================================================
\chapter{Création de fichier Python}

\begin{EExample*}{}{}
\begin{lstlisting}[style=PythonDraculaWhite]
	print("Hello World") 
\end{lstlisting}
\end{EExample*}
\noindent La fonction \entouree[gray]{\texttt{\footnotesize{print()}}} prend un l'argument entre parenthèse et 
\textit{l'affiche en ligne de commande}. On enregistre un fichier avec l'extension 
\entouree[gray]{\texttt{\footnotesize{.py}}}. Pour lancer un programma Python, on navigue verse le répertoire
contenant le fichier et on tape \entouree[gray]{\texttt{\footnotesize{python nomFichier.py}}}. Lorsqu'on lance cette commande, un interprète lie le fichier ligne par ligne et l'exécute en temps réel.  
%====================================================================
% 
%====================================================================
\chapter{Variables et types}
\begin{Syntaxe*}{}{}
La création de variable suit la syntaxe suivante : \entouree[gray]{\texttt{\footnotesize{nomVariable = valeur}}}
\end{Syntaxe*}

\begin{EExample*}{Création de varibles}{}
\begin{lstlisting}[style=PythonDraculaWhite]
	a = 28
    b = 1.5
    c = "Hello!"
    d = True
    e = None
\end{lstlisting}
\end{EExample*}

\begin{table}[H]
		\caption {Principaux types primtifs Python}

	\begin{center}
		\begin{tabular}{l|l}

	\arrayrulecolor{blue}\hline
	\rowcolor{lightBlue}
	\textcolor{myb}{\bfseries\fontfamily{lmss}\selectfont{Type}} & 
	\textcolor{myb}{\bfseries\fontfamily{lmss}\selectfont{Description}}
	\\
	\hline
	\arrayrulecolor{black}
	\texttt{\footnotesize{int}} & \footnotesize\textit{\texttt{Un entier}}
	\\
	\hline
	\texttt{\footnotesize{float}} & \footnotesize\textit{\texttt{Un nombre décimal}}
	\\
	\hline
	\texttt{\footnotesize{chr}} & \footnotesize\textit{\texttt{A simple caractère}}
	\\
	\hline
	\hline
	\texttt{\footnotesize{str}} & \footnotesize\textit{\texttt{Une chaîne de caractère}}
	\\
	\hline	
	\hline
	\texttt{\footnotesize{bool}} & 
	\footnotesize\textit{\texttt{Une valeur qui est soit}} \entouree[gray]{\texttt{\footnotesize{vrai}}}
	\footnotesize\textit{\texttt{ou}} \entouree[gray]{\texttt{\footnotesize{fausse}}} 
	\\
	\hline	
	\hline
	\texttt{\footnotesize{chr}} & \footnotesize\textit{\texttt{A simple caractère}}
	\\
	\hline	

	\end{tabular}
	\end{center}
\end{table}




\begin{EExample*}{Entrée d'utilisateur}{}
\begin{lstlisting}[style=PythonDraculaWhite]
	# Ce programme prends comme entree le nom de l'utilisateur et l'imprime en ligne de commande
	name = input("Name: ")
    print("Hello, " + name)
	# On utilise la fonction input() pour enregistre l'entree
\end{lstlisting}
\end{EExample*}
%====================================================================
% 
%====================================================================


\section{Formater une chaîne de caractère}
\begin{Syntaxe*}{Générer des \textit{fString}}{}
	On utilise la syntaxe \entouree[gray]{\texttt{\footnotesize{f"Expression \{variable\}" }}} pour concaténer une chaîne de caractère existante à la chaîne correspondante d'une variable.
\end{Syntaxe*}

\begin{EExample*}{Concaténation par \textit{fString}}{}
\begin{lstlisting}[style=PythonDraculaWhite]
	# Methode f string.
	print(f"Hello, {input("Name: ")}")

	# Methode classique
	print("Hello, " + name)
\end{lstlisting}
\end{EExample*}



\section{Conditions}
\begin{Syntaxe*}{Assertions conditionnelles}{}
Elles sont introduites par les mots clés \entouree[gray]{\texttt{\footnotesize{if}}} 
\entouree[gray]{\texttt{\footnotesize{elif}}} \entouree[gray]{\texttt{\footnotesize{else}}} suivit de deux points
\entouree[gray]{\texttt{\footnotesize{:}}}. L'indentation indique que \textit{l'expression suivante sera évaluée
à} \entouree[gray]{\texttt{\footnotesize{True}}} ou \entouree[gray]{\texttt{\footnotesize{False}}}
\end{Syntaxe*}

\begin{EExample*}{Assertions conditionnelles}{}
	\begin{lstlisting}[style=PythonDraculaWhite]
# Si on n'avait pas convertit l'entree en entier, l'interpretteur aurait lance une exception. 	
num = int(input("Number: "))
if num > 0:
    print("Number is positive")
elif num < 0:
    print("Number is negative")
else:
    print("Number is 0")
	\end{lstlisting}
\end{EExample*}


\section{Séquences}
\begin{DefG*}{Séquence Mutable}{}
	On dit qu'une séquence est mutable lorsqu'il est possible de changer les éléments individuels de cette 
	\textit{séquence} après sa création. 
\end{DefG*}
\begin{DefG*}{Séquence ordonnée}{}
	Séquence dans laquelle l'ordre des éléments est important. 
\end{DefG*}


\begin{note}
Les chaîne de caractères sont \textbf{ordonnées} et \textbf{immuables}
\end{note}
\begin{EExample*}{Accéder à un caractère}{}
	\begin{lstlisting}[style=PythonDraculaWhite]
		# On peut acceder a un caractere en specifiant la position de celui-ci par rapport la variable
		name = "Harry"
        print(name[0]) # Renvoit "H"
        print(name[1]) # Renvoit "a"

	\end{lstlisting}
\end{EExample*}


\section{Listes}
\begin{DefG*}{Liste Python}{}
	Structure de données qui nous permet \textit{d'enregistrer n'importe quel type}. Une liste est contenu par \entouree[gray]{\texttt{\footnotesize{[...]}}}. 
\end{DefG*}

\begin{EExample*}{Création de liste}{}
\begin{lstlisting}[style=PythonDraculaWhite]

names = ["Harry", "Ron", "Hermione"]
# Imprime la liste entiere 
print(names)
# Imprime le second element de la liste
print(names[1])
# Ajoute un nouveau nom a la liste
names.append("Draco")
# Organise la liste en ordre alphabetique
names.sort()
# Imprime la nouvelle liste
print(names)
\end{lstlisting}
\end{EExample*}
\begin{note}
Les listes sont \textbf{ordonnées} et \textbf{mutables}.
\end{note}



\section{Tuples}
\begin{DefG*}{Tuples}{}
	Structure de données généralement utilisée pour enregistrer \textit{deux ou trois types de valeurs}—des coordonnées par exemple.
\end{DefG*}
\begin{EExample*}{Création d'un tuple}{}
\begin{lstlisting}[style=PythonDraculaWhite]
# On l'utilise ici pour creer un systeme de coordonees 
point = (12.5, 10.6)
\end{lstlisting}
\end{EExample*}
\begin{note}
Les tuples sont \textbf{ordonnés} et \textbf{immuables}.
\end{note}




\section{Ensembles}	
\begin{DefG*}{Ensemble}{}
	Un ensemble est un types où une valeur \textit{ne peut être enregistré plus d'une fois}.
\end{DefG*}

\begin{EExample*}{Manipulation des ensembles}{}
\begin{lstlisting}[style=PythonDraculaWhite]
# Cree un ensemble vide
s = set()

# Ajoute des elements dans l'ensemble via la fonction add().
s.add(1)
s.add(2)
s.add(3)
s.add(4)
s.add(3)
s.add(1)

# Retire la valeur 2 de l'ensemble.
s.remove(2)

# Imprime l'ensemble
print(s)

# Determine et affiche la taille de l'ensemble
print(f"The set has {len(s)} elements.")

""" Ceci est un commentaire de plusieurs lignes en Python
Output:
{1, 3, 4}
The set has 3 elements.
"""
\end{lstlisting}
\end{EExample*}


\section{Dictionnaires}
\begin{DefG*}{Dictionnaire}{}
Ensemble de \entouree[gray]{\texttt{\footnotesize{clé:valeur}}} ; pour chaque clée, il y a une valeur correspondante. 
\end{DefG*}
\begin{codeEx*}{Création de dictionnaire}{}
\begin{lstlisting}[style=PythonDraculaWhite]
# Definit un dictionnaire
houses = {"Harry": "Gryffindor", "Draco": "Slytherin"}

# Imprime la maison a laquellle Harry apparient. 
print(houses["Harry"])

# Ajoute une valeur au dictionnaire. 
houses["Hermione"] = "Gryffindor"

# Imprime la maison a laquelle Hermione appartient.  
print(houses["Hermione"])

""" Output:
Gryffindor
Gryffindor
"""
\end{lstlisting}
\end{codeEx*}
\begin{note}
Les dictionnaires \entouree[gray]{\texttt{\footnotesize{non ordonnées}}} et \entouree[gray]{\texttt{\footnotesize{mutables}}}
\end{note}


\section{Boucles}
\begin{DefG*}{}{}
Utilisé pour itérer sur une séquence d'élément en appliquant un bloc d'instructions. 
\end{DefG*}

\begin{EExample*}{Utilisation de boucle for}{}
\begin{lstlisting}[style=PythonDraculaWhite]
# Imprime les nombres de 0 a 5. 
for i in [0, 1, 2, 3, 4, 5]:
    print(i)

""" Output:
0
1
2
3
4
5
"""
# Alternative en utilisant range() pour alleger l'ecriture
for i in range(6):
    print(i)

""" Output:
0
1
2
3
4
5
"""
# Cree une liste
names = ["Harry", "Ron", "Hermione"]

# Imprime chaque nom de la liste
for name in names:
    print(name)

""" Output:
Harry
Ron
Hermione
"""

Iterer sur une chaine de caractere :
name = "Harry"
for char in name:
    print(char)

""" Output:
H
a
r
r
y
"""
\end{lstlisting}
\end{EExample*}

\section{Fonctions}
\begin{EExample*}{Création de fonction}{}
\begin{lstlisting}[style=PythonDraculaWhite]
# Cette fonction prend comme argument un nombre et retourne son carre
def square(x):
    return x * x

for i in range(10):
    print(f"The square of {i} is {square(i)}")

""" Output:
The square of 0 is 0
The square of 1 is 1
The square of 2 is 4
The square of 3 is 9
The square of 4 is 16
The square of 5 is 25
The square of 6 is 36
The square of 7 is 49
The square of 8 is 64
The square of 9 is 81
"""
\end{lstlisting}
\end{EExample*}

\section{Modules}
\begin{note}
Il est possible d'importer des fonctions contenu dans un fichier pour les utiliser dans un autre
\end{note}

\begin{EExample*}{Importer une fonction}{}
\begin{lstlisting}[style=PythonDraculaWhite]
# Contenu du fichier functions.py
def square(x):
    return x * x
# ===============================================================================
# Contenu fu fichier square.py

from functions import square # Importe la fonction square du fichier functions.py
for i in range(10):
    print(f"The square of {i} is {square(i)}")

# ===============================================================================
# Alternative

import functions # On importe le module au complet 
for i in range(10):
    print(f"The square of {i} is {functions.square(i)}")
\end{lstlisting}
\end{EExample*}

\section{Programmation Orientée Objet}
\begin{DefG*}{}{}
Il s'agit d'un paradigme de programmation qui met de l'avant les objets qui peuvent 
\textit{enregistrer de l'information et effectuer des actions}.
\end{DefG*}
\begin{DefG*}{Classes Python}{}
Ce sont des types définit par le programmeur. C'est essentiellement le patron pour un nouveau
type d'objet qui peut enregstrer de l'information et effectuer des actions. 
\end{DefG*}

\begin{codeEx*}{Créer une classe}{}
\begin{lstlisting}[style=PythonDraculaWhite]
class Point():
    # A method defining how to create a point:
    def __init__(self, x, y):
        self.x = x
        self.y = y

# Utilisation de la classe Point()
p = Point(2, 8)
print(p.x)
print(p.y)

""" Output:
2
8
"""
# =================================================================================
class Flight():
    # Methode qui cree un nouveau vol avec un capacite donnee
    def __init__(self, capacity):
        self.capacity = capacity
        self.passengers = []

    # Method qui ajoute un passager au vol
    def add_passenger(self, name):
        if not self.open_seats():
            return False
        self.passengers.append(name)
        return True

    # Methode qui retourne le nombre de places disponibles
    def open_seats(self):
        return self.capacity - len(self.passengers)
# =================================================================================
# Utilisation de la classe Flight()

# Cree un nouveau vol avec capacite de 3 passagers
flight = Flight(3)

# Cree une liste de personnes
people = ["Harry", "Ron", "Hermione", "Ginny"]

# Tente d'ajoute chaque personne de la liste au vol
for person in people:
    if flight.add_passenger(person):
        print(f"Added {person} to flight successfully")
    else:
        print(f"No available seats for {person}")

""" Output:
Added Harry to flight successfully
Added Ron to flight successfully
Added Hermione to flight successfully
No available seats for Ginny
"""
\end{lstlisting}
\end{codeEx*}

\section{Programmation fonctionnelle}
\begin{DefG*}{Programmation fonctionnelle}{}
Paradigme de programmation où on traite chaque fonction comme n'importe quelle autre variable.
\end{DefG*}
\begin{EExample*}{}{}
\begin{lstlisting}[style=PythonDraculaWhite]
# On peut simplifier l'ecriture d'une fonction courte grace a lambda
square = lambda x: x * x
# =================================================================================
# Organiser une liste de dictionnaire
people = [
{"name": "Harry", "house": "Gryffindor"},
{"name": "Cho", "house": "Ravenclaw"},
{"name": "Draco", "house": "Slytherin"}
]

def f(person):
    return person["name"]

people.sort(key=f)

print(people)

""" Output:
[{'name': 'Cho', 'house': 'Ravenclaw'}, {'name': 'Draco', 'house': 'Slytherin'}, {'name': 'Harry', 'house': 'Gryffindor'}]
"""
# =================================================================================
# Methode avec fonction lamda

people = [
    {"name": "Harry", "house": "Gryffindor"},
    {"name": "Cho", "house": "Ravenclaw"},
    {"name": "Draco", "house": "Slytherin"}
]

people.sort(key=lambda person: person["name"])

print(people)

""" Output:
[{'name': 'Cho', 'house': 'Ravenclaw'}, {'name': 'Draco', 'house': 'Slytherin'}, {'name': 'Harry', 'house': 'Gryffindor'}]
"""
\end{lstlisting}
\end{EExample*}
%====================================================================
% 
%====================================================================
\section{Exceptions}
\begin{Syntaxe*}{Gestion des exceptions}{}
Il arrive qu'on effectue des opérations qui lèvent une exception. On peut gérer ces instances avec un les commandes \entouree[gray]{\texttt{\footnotesize{try}}} et \entouree[gray]{\texttt{\footnotesize{excpet}}} 
\end{Syntaxe*}
\begin{EExample*}{Division par zéro}{}
\begin{lstlisting}[style=PythonDraculaWhite]
# Ce programme risque de lever une exception si le diviseur (y) est zero
x = int(input("x: "))
y = int(input("y: "))

result = x / y
# =================================================================================

# L'alternative est un programme incorporant try et except.
import sys

x = int(input("x: "))
y = int(input("y: "))

try:
    result = x / y
except ZeroDivisionError:
    print("Error: Cannot divide by 0.")
    # Exit the program
    sys.exit(1)

print(f"{x} / {y} = {result}")



print(f"{x} / {y} = {result}")
\end{lstlisting}
\end{EExample*}



\end{document}



